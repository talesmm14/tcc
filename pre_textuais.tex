% ----------------------------------------------------------
% ELEMENTOS PRÉ-TEXTUAIS
% ----------------------------------------------------------
% \pretextual

% ---
% Capa
% ---
\imprimircapa
% ---

% ---
% Folha de rosto
% (o * indica que haverá a ficha bibliográfica)
% ---
\imprimirfolhaderosto
% ---

% ---
% Inserir folha de aprovação
% ---

% Isto é um exemplo de Folha de aprovação, elemento obrigatório da NBR
% 14724/2011 (seção 4.2.1.3). Você pode utilizar este modelo até a aprovação
% do trabalho. Após isso, substitua todo o conteúdo deste arquivo por uma
% imagem da página assinada pela banca com o comando abaixo:
%
% \includepdf{folhadeaprovacao_final.pdf}
%
\begin{folhadeaprovacao}



	\begin{center}
		\includegraphics[width=1\textwidth]{imagens/unitins.png}
		\ABNTEXchapterfont\Large   CURSO DE SISTEMAS DE INFORMA{\c{C}}{\~{A}}O

		\par
		\vspace*{1cm}
		{\ABNTEXchapterfont\bfseries\large \expandafter\MakeUppercase{\imprimirtitulo}  \vspace*{1cm}    }
		\par
		{\large \expandafter\MakeUppercase{\imprimirautor}}
		%\vspace*{\fill}
		\par
		\vspace*{1cm}
		\hspace{.45\textwidth}
		\begin{minipage}{.5\textwidth}
			\small\imprimirpreambulo

		\end{minipage}%
		%	\vspace*{\fill}
	\end{center}



	\assinatura{\textbf{\imprimirorientador} \\ Orientador}
	\assinatura{\textbf{Professor} \\ Convidado 1}
	\assinatura{\textbf{Professor} \\ Convidado 2}
	%\assinatura{\textbf{Professor} \\ Convidado 3}
	%\assinatura{\textbf{Professor} \\ Convidado 4}

	\begin{center}
		\vspace*{0.5cm}
		{\large\imprimirlocal}
		\par
		{\large\imprimirdata}
		\vspace*{1cm}
	\end{center}

\end{folhadeaprovacao}
% ---

% ---
% Dedicatória
% ---
\begin{dedicatoria}
	\vspace*{\fill}
	\centering
	\noindent
	\textit{A todas as pessoas que direta ou indiretamente contribuíram para a realização deste trabalho.} \vspace*{\fill}
\end{dedicatoria}
% ---

% ---
% Agradecimentos
% --- Es
\begin{agradecimentos}
	Ao meu orientador, pelo suporte no pouco tempo que lhe coube, pelas suas correções e incentivos.
	Agradeço a todos os  meus professores por me proporcionar o conhecimento e afetividade da educação 
	no processo de formação profissional, pela excelência da qualidade técnica de cada um.
	Aos meus pais que sempre me incentivaram a superar as dificuldades.
	Aos meus amigos de jornada, por não me deixarem desistir.
	A todos que direta ou indiretamente fizeram parte de minha formação, o meu muito obrigado.


\end{agradecimentos}
% ---

% ---
% Epígrafe
% ---
\begin{epigrafe}
	\vspace*{\fill}
	\begin{flushright}
		\textit{``The way to the stars is open.''\\
			Sergei Korolev}
	\end{flushright}
\end{epigrafe}
% ---

% ---
% RESUMOS
% ---

% resumo em português
\setlength{\absparsep}{18pt} % ajusta o espaçamento dos parágrafos do resumo
\begin{resumo}
	A leitura de dados vindos de sensores em tempo real sempre foi uma tarefa difícil e custosa. Há uma grande quantidade de ruído proveniente da baixa qualidade dos sensores e sequelas resultantes do ambiente, que produzem uma imensa quantidade de dados imprecisos e incorretos, estes os quais atrapalham na tomada de decisão de sistemas embarcados. 
	% O problema é muito estudado pela comunidade científica, diversos trabalhos apontam diferentes formas de tratar e ignorar ruídos advindos de sensores. Notou-se a grande presença de artigos recentes que aprofundam o assunto em diversas áreas do conhecimento onde a leitura de dados tem um papel crítico e decisivo no resultado final, os quais são tratados e apresentados aqui com uma revisão bibliométrica que apresenta uma visão dos termos, artigos e autores dos últimos 5 anos desde a concepção deste trabalho. 
	Neste trabalho foi implementado uma função de intervalo de confiança móvel, que pode ser utilizada em conjunto com outros métodos para obter dados mais precisos sem interferir no resultado final, o código tem como objetivo trabalhar de forma amigável em ambiente de sistema embarcado onde memória e processamento são recursos escassos e valiosos. Os resultados mostraram que o intervalo de confiança pode ser utilizado em certas ocasiões onde o tempo de espera não é tão custoso, retornando dados sempre dentro de um padrão aceitável de leitura mas perdendo qualidade na leitura de grandes curvas.
	
	% Neste trabalho também foi implementado funções de filtro populares e uma função de filtragem probabilística, em um repositório de código aberto apelidado de zscilib, focado em um conjunto de funções úteis para computação científica, análise de dados e manipulação de dados no contexto de dispositivos de hardware embarcado, desenvolvido especialmente para o sistema operacional de tempo real Zephyr mantido pela fundação Linux. As contribuições aqui propostas serão integradas a comunidade de código aberto e servirão para que outros pesquisadores, estudantes e atuantes da área possam usufruir e contribuir com a implementação código proposto.



	\textbf{Palavras-chaves}: Algoritmo de filtragem, Sistemas embarcados, Software, Sensor, Redução de ruído.
\end{resumo}

% resumo em inglês
\begin{resumo}[Abstract]
	\begin{otherlanguage*}{english}
		...
		\vspace{\onelineskip}

		\noindent
		\textbf{Key-words}: Filtering algorithm, Embedded systems, Software, Sensor, Noise reduction.
	\end{otherlanguage*}
\end{resumo}


% ---
% inserir lista de ilustrações
% ---
\pdfbookmark[0]{\listfigurename}{lof}
\listoffigures*
% ---

% ---
% inserir lista de tabelas
% ---
\newpage
\pdfbookmark[0]{\listtablename}{lot}
\newpage
\listoftables*
\cleardoublepage
% ---

% ---
% inserir lista de abreviaturas e siglas
% ---
\begin{siglas}
	\item RTOS - Sistema operacional de tempo-real.
	\item ISR - Rotinas de serviço de interrupção.
	\item SO - Sistema operacional.
	\item IC - Intervalo de confiança.
	\item SRAM - Memória Estática de Acesso Aleatório
	\item ROM - Memória de somente leitura
\end{siglas}
% ---


% ---
% inserir o sumario
% ---
\pdfbookmark[0]{\contentsname}{toc}
\tableofcontents*
\cleardoublepage
% ---

