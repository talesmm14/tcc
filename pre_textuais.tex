% ----------------------------------------------------------
% ELEMENTOS PRÉ-TEXTUAIS
% ----------------------------------------------------------
% \pretextual

% ---
% Capa
% ---
\imprimircapa
% ---

% ---
% Folha de rosto
% (o * indica que haverá a ficha bibliográfica)
% ---
\imprimirfolhaderosto
% ---

% ---
% Inserir folha de aprovação
% ---

% Isto é um exemplo de Folha de aprovação, elemento obrigatório da NBR
% 14724/2011 (seção 4.2.1.3). Você pode utilizar este modelo até a aprovação
% do trabalho. Após isso, substitua todo o conteúdo deste arquivo por uma
% imagem da página assinada pela banca com o comando abaixo:
%
% \includepdf{folhadeaprovacao_final.pdf}
%
\begin{folhadeaprovacao}



	\begin{center}
		\includegraphics[width=1\textwidth]{imagens/unitins.png}
		\ABNTEXchapterfont\Large   CURSO DE SISTEMAS DE INFORMA{\c{C}}{\~{A}}O

		\par
		\vspace*{1cm}
		{\ABNTEXchapterfont\bfseries\large \expandafter\MakeUppercase{\imprimirtitulo}  \vspace*{1cm}    }
		\par
		{\large \expandafter\MakeUppercase{\imprimirautor}}
		%\vspace*{\fill}
		\par
		\vspace*{1cm}
		\hspace{.45\textwidth}
		\begin{minipage}{.5\textwidth}
			\small\imprimirpreambulo

		\end{minipage}%
		%	\vspace*{\fill}
	\end{center}



	\assinatura{\textbf{\imprimirorientador} \\ Orientador}
	\assinatura{\textbf{Professor} \\ Convidado 1}
	\assinatura{\textbf{Professor} \\ Convidado 2}
	%\assinatura{\textbf{Professor} \\ Convidado 3}
	%\assinatura{\textbf{Professor} \\ Convidado 4}

	\begin{center}
		\vspace*{0.5cm}
		{\large\imprimirlocal}
		\par
		{\large\imprimirdata}
		\vspace*{1cm}
	\end{center}

\end{folhadeaprovacao}
% ---

% ---
% Dedicatória
% ---
\begin{dedicatoria}
	\vspace*{\fill}
	\centering
	\noindent
	\textit{A todas as pessoas que direta ou indiretamente contribuíram para a realização deste trabalho.} \vspace*{\fill}
\end{dedicatoria}
% ---

% ---
% Agradecimentos
% --- Es
\begin{agradecimentos}
	Ao meu orientador, pelo suporte no pouco tempo que lhe coube, pelas suas correções e incentivos.
	Agradeço a todos os  meus professores por me proporcionar o conhecimento e afetividade da educação 
	no processo de formação profissional, pela excelência da qualidade técnica de cada um.
	Aos meus pais que sempre me incentivaram a superar as dificuldades.
	Aos meus amigos de jornada, por não me deixarem desistir.
	A todos que direta ou indiretamente fizeram parte de minha formação, o meu muito obrigado.


\end{agradecimentos}
% ---

% ---
% Epígrafe
% ---
\begin{epigrafe}
	\vspace*{\fill}
	\begin{flushright}
		\textit{``The way to the stars is open.''\\
			Sergei Korolev}
	\end{flushright}
\end{epigrafe}
% ---

% ---
% RESUMOS
% ---

% resumo em português
\setlength{\absparsep}{18pt} % ajusta o espaçamento dos parágrafos do resumo
\begin{resumo}
	O desenvolvimento de satélites é explorado por empresas, entusiastas e universidades em função da 
	redução acentuada de custos, disponibilidade tecnológica e interesse em aprimoramento científico. 
	Atualmente, padrões para a construção de espaçonaves de dimensões reduzidas em formato cúbico fazem 
	parte do cotidiano de estudantes. Com a utilização de componentes de prateleira e sistemas 
	operacionais de código aberto exploram o ambiente científico voltado para o espaço. Os CubeSats são 
	um desafio de engenharia de sistemas com o desenvolvimento de firmware e software para este tipo 
	dispositivo dotado de severos requisitos relacionados às propriedades importantes como precisão, 
	desempenho e determinismo. Dentre os componentes de prateleira utilizados na prototipação e implantação, 
	se destaca o microcontrolador ESP32, o qual possui baixo consumo de energia e alta performance. 
	Os projetos de CubeSats de código aberto buscam um sistema operacional por sua robustez, recorrendo ao 
	FreeRTOS já validado em diversas missões, entretanto, atrelado ao potencial 
	do microcontrolador, avaliar a utilização de outros sistemas de código aberto é um objetivo promissor. 
	O projeto Zephy é um Sistema Operacional de Tempo Real da Fundação Linux sob licença Apache 2.0 que possui 
	suporte às arquiteturas embarcadas utilizadas em plataformas de microssatélites. Desta forma, este trabalho 
	propõe avaliar a precisão, desempenho em tarefas rotineiras da operação de uma missão espacial, estabelecer 
	conjunto de testes de benchmark para comparar os dois sistemas operacionais. Os diversos benefícios gerados
	para a comunidade são, além de oferecer uma nova ferramenta para o mercado, métricas de benchmark que
	através deste trabalho sejam desenvolvidas outras pesquisas complementando táticas de se avaliar sistemas 
	para missões espaciais.



	\textbf{Palavras-chaves}: Embedded Kernel, Onboard computer, Software, CubeSat, RTOS, Zephyr.
\end{resumo}

% resumo em inglês
\begin{resumo}[Abstract]
	\begin{otherlanguage*}{english}
With the democratization of access to space, the development of satellites increasingly becomes 
accessible to the common public such as companies, enthusiasts and universities, with different standards emerging 
for building spaceships with off-the-shelf components and code operating systems 
open. Firmware and software development for this type of device needs requirements 
strings related to important properties such as accuracy, performance, and determinism. This one 
work aims to evaluate the Zephyr real-time operating system on the ESP32 platform, 
as an alternative to RTOS to be used in the construction of CubeSats, whose benefits of 
using an realtime operating system are numerous, in this work a set of benchmark tests are presented. 
		\vspace{\onelineskip}

		\noindent
		\textbf{Key-words}: Embedded Kernel, Onboard computer, Software, CubeSat, RTOS, Zephyr.
	\end{otherlanguage*}
\end{resumo}


% ---
% inserir lista de ilustrações
% ---
\pdfbookmark[0]{\listfigurename}{lof}
\listoffigures*
% ---

% ---
% inserir lista de tabelas
% ---
\newpage
\pdfbookmark[0]{\listtablename}{lot}
\newpage
\listoftables*
\cleardoublepage
% ---

% ---
% inserir lista de abreviaturas e siglas
% ---
\begin{siglas}
	\item RTOS - Sistema operacional de tempo-real.
	\item ISR - Interrupt Service Routines.
	\item SO - Sistema operacional.
	\item GPIO - General Purpose Input/Output.
	\item COTS - Commercial Off The Shelf
	\item LEO - Low Earth Orbit
	\item UFSM - Universidade de Santa Maria
	\item INPE - Instituto Nacional de Pesquisas Espaciais
	\item MCTI - Ministério de Ciência e Tecnologia
	\item AEB - Agência Espacial Brasileira
	\item IDE - Ambiente de desenvolvimento integrado
\end{siglas}
% ---


% ---
% inserir o sumario
% ---
\pdfbookmark[0]{\contentsname}{toc}
\tableofcontents*
\cleardoublepage
% ---

