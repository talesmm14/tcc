
\chapter{Conclusão}\label{cap:conclusao}


O mercado de pequenos satélites e de extrema importância para qualquer pais, 
universidade ou empresa que deseja entrar no ramo espacial, já que o mesmo 
representa uma grande parcela dos lançamentos com um custo relativamente baixo 
mas com um grande potencial de avanço tecnológico.

Esperasse que o Zephyr seja capaz de suportar uma missão de CubeSat completa 
já que o mesmo é bastante completo. Os testes aqui presentes buscam validar se seu 
tempo de resposta e compatível com outro RTOS já usado no mercado, com essa 
contribuição aguardamos que as equipes possam ter outra oferta de peso para 
desenvolvimento, com uma comunidade grande e com possibilidade de ser ainda 
maior, como também provando a qualidade do processador ESP32 em ser versátil 
a qualquer sistema operacional do mercado, com sua rápida adoção pela comunidade. 
Os testes deste trabalho não devem ser usados exclusivamente para 
a escolha correta de um RTOS, já que se deve levar em consideração muitos 
outros fatores não abordados neste trabalho, já que projetos de espaçonaves 
aderem muitos outros campos, não sendo exclusividade somente do tempo como 
uma variável exclusivamente importante. 
Também com o estudo gerado, esperasse que o mesmo seja usado em outras 
plataformas fora o ESP32, atribuindo métricas importantes para a escolha de um 
RTOS novo em uma missão. 

Missões espaciais tem muitos critérios, este trabalho deseja ser o ponta pé em 
outros trabalhos que também possam avaliar e definir novas métricas de escolha, 
que facilitem e deem tempo a equipes de construção, já que o principal objetivo 
de um pequeno satélite e ser barato, seu tempo de desenvolvimento conta muito em 
seu custo final.


