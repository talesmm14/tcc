
\chapter{Conclusão}\label{cap:conclusao}

Os sensores fazem parte da revolução tecnológica que o mundo tem passado, os dados coletados pelos mesmos são utilizados nas mais diversas esferas, são relevantes para as tomadas de decisão e estão sempre sujeitos a interferências.
 
Espera-se que todo sinal vindo de qualquer sensor esteja com algum ruído, e que sejam tratados de alguma forma. Para isso existem diversas funções para se retirar valores discrepantes, muitas delas interferem na forma do sinal real tratando o dado e retirando um novo valor resultante e de maior precisão, essa ação pode ser prejudicial em muitos dos casos alterando o real valor e dificultando ou atrasando a visualização de mudanças bruscas, muito importantes em algumas analise de dados. 
No ambiente de sistemas embarcados é muito importante que qualquer algoritmo respeite as limitações de memória, processamento e tempo. Algumas funções podem exigir muito processamento ou até mesmo muita memória para realizar todos os cálculos responsáveis pela filtragem, com esse problema é importante que a função esteja dentro dos parâmetros para trabalhar em ambiente restrito.

Este trabalho teve como objetivo apresentar o desvio de confiança como uma boa alternativa para ser utilizada como filtro em ambiente embarcado, além disso é importante que os dados filtrados permaneçam os mais originais possíveis. 
O resultado encontrado com o algoritmo proposto é capaz de manipular um vetor de amostra móvel com eficácia, além de manter os dados de forma original mas com o crivo dos valores dentro de um intervalo aceitável de confiança. 
Mas observou-se problemas importantes, como não conseguir registrar sinais em curvas muito prolongadas, o que tornava a espera por um novo sinal confiável muito longa e atrapalhava o andamento do programa, por mais que o algoritmo tenha se saído bem em retornar os valores superiores de todos os picos de valores. Também notou-se que os dados resultantes por serem intocáveis, entregavam a possibilidade do método ser utilizado em conjunto com outros que posteriormente coletariam os dados, filtrando-os em busca de valores mais amenizados dependendo da aplicação.
 
O trabalho aqui desenvolvido faltou com a realização de todos os cálculos com outros tipos de sensores, o tratamento provou-se ser viável em ambiente controlado e poderia ser aplicado em conjunto com outros métodos em trabalhos futuros. Retornando à comunidade de desenvolvimento de filtros digitais em sistemas embarcados uma alternativa aberta a ser utilizada.


\section{Trabalhos Futuros}

\begin{itemize}
    \item Utilizar o filtro apresentado com outras funções de filtragem.
    \item Escrever o filtro em uma aplicação real de coleta de dados.
  \end{itemize}
