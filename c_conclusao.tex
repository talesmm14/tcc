
\chapter{Conclusão}\label{cap:conclusao}

Os sensores fazem parte da revolução tecnológica que o mundo tem passado, os dados coletados pelos mesmos são utilizados nas mais diversas esferas, são importantíssimos para as tomadas de decisão e estão sempre sujeitos a interferências.
 
Espera-se que todo sinal vindo de qualquer sensor esteja com algum ruído, e que sejam tratados de alguma forma. Para isso existem diversas funções para se retirar valores discrepantes, muitas delas interferem na forma do sinal real tratando o dado e retirando um novo valor resultante e de maior precisão, essa ação pode ser prejudicial em muitos do casos alterando o real valor e dificultando ou atrasando a visualização de mudanças bruscas, muito importantes em algumas analise de dados. No ambiente de sistemas embarcados é muito importante que qualquer algoritmo respeite as limitações de memória, processamento e tempo. Algumas funções podem exigir muito processamento ou até mesmo muita memória para realizar todos os cálculos responsáveis pela filtragem, com esse problema é importante que a função esteja dentro dos parâmetros para trabalhar em ambiente restrito.
Este trabalho teve como objetivo apresentar o desvio de confiança como uma boa alternativa para ser utilizada como filtro em ambiente embarcado, além disso era importante que os dados filtrados fossem os mais originais possíveis. O resultado foi um algoritmo que trabalhou bem com um pequeno vetor de amostra móvel, que manteve os dados de forma original mas com o crivo dos valores dentro de um intervalo aceitável de confiança. Mas observou-se problemas importantes, como não conseguir registrar sinais em curvas muito prolongadas, o que tornava a espera por um novo sinal confiável muito longa e atrapalhava o andamento do programa, por mais que o algoritmo tenha se saído bem em retornar os valores superiores de todos os picos de valores. Também notou-se que os dados resultantes por serem intocáveis, entregavam a possibilidade do método ser utilizado em conjunto com outros que posteriormente coletariam os dados, filtrando-os em busca de valores mais amenizados dependendo da aplicação.
 
O tempo envolvido neste trabalho não foi o suficiente para a realização de todos os cálculos em linguagem C em um sistema embarcado, mas o tratamento provou-se ser viável e poderia ser aplicado em conjunto com outros métodos em trabalhos futuros. Retornando à comunidade de desenvolvimento de filtros digitais em sistemas embarcados uma alternativa aberta a ser utilizada.




% O mercado de pequenos satélites e de extrema importância para qualquer pais, 
% universidade ou empresa que deseja entrar no ramo espacial, já que o mesmo 
% representa uma grande parcela dos lançamentos com um custo relativamente baixo 
% mas com um grande potencial de avanço tecnológico.

% Esperasse que o Zephyr seja capaz de suportar uma missão de CubeSat completa 
% já que o mesmo é bastante completo. Os testes aqui presentes buscam validar se seu 
% tempo de resposta e compatível com outro RTOS já usado no mercado, com essa 
% contribuição aguardamos que as equipes possam ter outra oferta de peso para 
% desenvolvimento, com uma comunidade grande e com possibilidade de ser ainda 
% maior, como também provando a qualidade do processador ESP32 em ser versátil 
% a qualquer sistema operacional do mercado, com sua rápida adoção pela comunidade. 
% Os testes deste trabalho não devem ser usados exclusivamente para 
% a escolha correta de um RTOS, já que se deve levar em consideração muitos 
% outros fatores não abordados neste trabalho, já que projetos de espaçonaves 
% aderem muitos outros campos, não sendo exclusividade somente do tempo como 
% uma variável exclusivamente importante. 
% Também com o estudo gerado, esperasse que o mesmo seja usado em outras 
% plataformas fora o ESP32, atribuindo métricas importantes para a escolha de um 
% RTOS novo em uma missão. 

% Missões espaciais tem muitos critérios, este trabalho deseja ser o ponta pé em 
% outros trabalhos que também possam avaliar e definir novas métricas de escolha, 
% que facilitem e deem tempo a equipes de construção, já que o principal objetivo 
% de um pequeno satélite e ser barato, seu tempo de desenvolvimento conta muito em 
% seu custo final.


\section{Trabalhos Futuros}

\begin{itemize}
    \item Utilizar o filtro apresentado com outras funções de filtragem.
    \item Escrever o filtro em uma aplicação real de coleta de dados.
  \end{itemize}
