% ---
% Capitulo de METODOLOGIA
% ---

\chapter{Definição de experimentos}\label{cap:definicoes}
\section{Introdução}
O experimento proposto neste trabalho seguirá da seguinte maneira, os testes serão
escritos na linguagem C nos dois RTOS escolhidos, ambos usarão funções iguais para
medida de tempo, cada teste terá seu arquivo individual para que não sofra
interferência de outros testes, todos serão disponibilizados na internet como código
aberto, afim que qualquer pessoa possa testar e validar os resultados aqui obtidos.

È muito importante que um sistema operacional de tempo real não deixe que tarefas
ocupem muito tempo para tomarem suas responsabilidades, isto impacta diretamente no
ciclo de toda a missão, tendo em vista que todo tempo e precioso para uma aplicação
que reaja quase em tempo real a qualquer situação. A medida de tempo será a subtração
do tempo de coleta inicial com o final, estas métricas são destinadas a oferecer
informações de tempo já que se supõe que o mesmo e muito importante para missões que
utilizem um RTOS.
% Definir o experimento
% Como vai ser feito
% Resultado esperado

\subsection{Velocidade de troca entre tarefas}
Durante o processo de encerramento de uma tarefa, o sistema operacional pode realizar
algumas operações, como vasculhar interrupções e quais as possíveis tarefas que
entrarão em seguida, tudo isso consome um tempo precioso para a aplicação
interferindo no tempo de troca para outra atividade, com o intuito de medir esse
tempo em milissegundos usaremos a função millis() para obter o tempo de troca entre
duas tarefas, subtraindo o tempo obtido no final da primeira tarefa e no começo da
segunda. Serão criadas duas tarefas,


%  Qual a politica de fila adotada?
uma com prioridade maior que a outra,


sendo a
com menor prioridade tendo a ação de despertar a outra, o tempo que leva para a outra
tarefa ser despertada será medido. Espera-se que a troca entre as tarefas seja a mais
rápida possível, não apresentando gargalos, e que a tarefa com menor prioridade consiga
despertar a outra tarefa, mesmo que seja de prioridade maior.

\subsection{Tempo de passagem de mensagens entre tarefas}
É muito importante que as tarefas possam se comunicar entre si, de preferência de
forma mais rápida possível evitando que se encontrem bloqueadas esperando alguma
resposta, o tempo de envio e recebimento será medido tendo entre tarefas com prioridades
iguais, quanto entre tarefas de prioridades diferentes, espera-se que o sistema evite
bloqueios após a solicitação de envio em uma mensagens. Durante uma missão as tarefas
trocaram muitas mensagens, já que muito da operação depende que os módulos do satélite
troquem informações entre si, a troca destas mensagens não pode prejudicar o andamento
de todo o sistema.

\subsection{Tempo de bloqueio e liberação de desbloqueio de Semáforo e Mutex}
Iremos medir o tempo que o RTOS leva para bloquear um Semáforo e Mutex, junto também
o tempo para liberação, este teste necessita apenas de uma tarefa, a qual solicita e
libera em seguida o semáforo ou mutex. È importante que este tempo seja o mais curto
possível, já que outras tarefas também podem depender desta mesma variável, sendo o
tempo de liberação mais importante. Durante uma missão varias

% Que tipo de operações serão realizadas?
tarefas

pediram permissão
para usar o mesmo recurso, sendo que o controle de acesso para evitar possíveis
conflitos responsabilidade dos Semáforos e Mutexes, pouco atraso na troca destes pode
ocasionar em possíveis problemas de prioridades, levando com que tarefas com prioridades
maiores levem mais tempo para conseguirem liberação.

\subsection{Tempo de resposta a eventos externos atraves de interrupção}
Durante uma missão o sistema lidara com diversas interrupções, sendo elas de alta
prioridade e devendo serem tratadas imediatamente, este teste esta muito
correlacionado com o primeiro, já que iremos medir o tempo que leva para uma tarefa
despertar através desta vez por uma

% Como será implementado essa rotina?
rotina de interrupção.

O controlador da missão
terá que lidar com diversas ISR durante a missão, sendo importante que nenhuma delas
fique para trás ou que atrapalhem o ciclo de vida do sistema.

\subsection{Velocidade de alocação de memoria}
Alocar memória para realizar alguma operação pode levar algum tempo, todo RTOS tem
suas funções para alocação de memória, e cabe a este teste coletar medida de tempo
nesta operação, seja qual for o tamanho de memória requisitada o sistema deve
entrega-la rapidamente, aqui selecionaremos uma região de memória RAM e o tempo para
aquisição e liberação da mesma será medida. Aqui se destaca cada RTOS com suas regras
de alocação de blocos de memória, poderemos notar qual a diferença das abordagens e se
adequam nas operações rápidas de uma tarefa.

\subsection{Velocidade de leitura e gravação em memoria flash (Cartão SD)}
Este teste é muito semelhante ao anterior, mas se difere por utilizar de memória flash,
aqui mais especifico duas memórias, já que faremos testes alocando memoria flash
internado ESP32 como também um cartão SD externo. Em muitas missões de CubeSat memoria
flash externas são usadas como backup do sistema entre outras aplicações como
armazenamentos de dados científicos antes do seu envio para a terra. Espera-se que
esta operação seja rápida e não corrompida, já que o envio destes dados e crucial para
o objetivo final da missão.
