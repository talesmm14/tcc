% ----------------------------------------------------------
% Introdução (exemplo de capítulo sem numeração, mas presente no Sumário)
% ----------------------------------------------------------

\chapter{Introdução}\label{intro}
\section{Conceitos Introdutórios e Problematização}

O mundo moderno é cercado por sensores passivos que coletam dados de praticamente tudo, para monitoramento ou tomada de decisão, a maioria desses dispositivos é bombardeada de interferências indesejáveis o tempo todo. Seja pela baixa qualidade dos equipamentos ou pela grande quantidade de intervenções aleatórias de fontes externas que podem prejudicar o resultado final.

Umas das abordagens mais populares para escapar de valores discrepantes, e a utilização do cálculo de média móvel ou média móvel ponderada para remover ruídos, no entanto essa abordagem não apresenta uma boa eficiência energética e demanda muito tempo para conseguir refletir mudanças bruscas no conjunto de dados utilizado \cite{International_Conference__Zhuang}. 
Os dados ruidosos se caracterizam como leituras perdidas dos sensores ou leituras imprecisas e não confiáveis, dificultando o trabalho das aplicações que tentam usar os valores colhidos em sua forma original \cite{Jeffery_Pipelined_Framework}, com isso quase todas as aplicações que tentam ser confiáveis utilizam algum tipo de algoritmo de filtragem de dados advindos de sensores eletrônicos.
Os sensores modernos são construídos de forma a serem pequenos, baratos e eficientes energeticamente, para que possam ser utilizados em todos os lugares de forma massiva e descartável, consequentemente a qualidade não é eficiente o suficiente para não apresentar problemas com interferências \cite{tan2005sensoclean}. 


O método de média interfere nos valores finais alterando ou não conseguindo acompanhar em tempo real características importantíssimas como picos e alterações bruscas no valor, é importante que o método de filtragem distorça o mínimo possível principalmente se tratar de dados que irão participar de tomadas de decisões importantes, outro principal problema na filtragem de dados e a falta de um critério quantitativo da qualidade do filtro \cite{kalambet2011noise}.





\section{Objetivos e Escopo de Pesquisa}
\subsection{Objetivos de Pesquisa}
O trabalho aqui proposto se dispõe em testar o método de intervalo de confiança, como uma alternativa aos métodos tradicionais para obtenção de dados provindos de sensores sem interferência de ruídos, visto a exorbitante quantidade de dados provindo de sensores, somando a incerteza da qualidade de fabricação dos componentes eletrônicos de baixo custo disponíveis em ampla quantidade no mercado, com dados coletados de sensores conectados ao kit de desenvolvimento ESP32 v1 com o sistema operacional de tempo real \cite{Zephyr}.

% Com isso em mente o trabalho aqui proposto se dispõem de construir uma biblioteca de código aberto, que disponibilizara funções adequadas para a filtragem de dados provindo de sensores em ambiente de tempo real com multitarefas, testando a coleta e filtragem de dados no processador dual core ESP32 no sistema operacional de tempo real \cite{Zephyr}.

\subsection{Objetivos secundários}
Os objetivos secundários a serem alcançados são:
\begin{itemize}
\item Realizar uma revisão bibliométrica de trabalhos que utilizam de métodos probabilísticos para filtragem de dados de sensores
\item Avaliar a utilização do método aqui proposto como uma alternativa aos métodos de média móvel e média móvel ponderada
% \item Disponibilizar o código de forma aberta a comunidade, para que possa ser utilizado por qualquer outro interessado em tratar dados de sensores em ambiente de multitarefas 
\item Escrever funções de filtragem utilizando os métodos de Desvio padrão e Intervalo de confiança
\end{itemize}


\section{Justificativa}
Dados provindos do mundo real são constantemente contaminados com ruídos ou podem não corresponder à realidade, esse tipo de situação pode afetar perigosamente um programa de computador qualquer que lida com dados de sensores. Muitos programas podem depender que essa resposta deva ser rápida o suficiente para não comprometer o desempenho e a segurança do programa, um veículo autômato ou um equipamento médico por exemplo não podem trabalhar com dados incorretos, mesmo que sejam em curtos períodos de tempo, os mesmos dependem que a resposta vinda dos sensores sejam rápidas e verdadeiras, com as respostas discrepantes sendo descartadas não comprometendo sua missão. 
Assim, em um primeiro momento considerou-se desenvolver e testar uma função de intervalo de confiança móvel para a filtragem de dados indesejáveis vindo de sensores diversos em tempo real, tendo como característica atuar de forma amigável junto a um RTOS sobre o microprocessador ESP32 da fabricante Espressif, em conjunto com o RTOS Zephyr um projeto mantido pela fundação Linux.
% Assim então este trabalho visa contribuir com a construção de uma biblioteca na linguagem C é de código aberto, onde será oferecido funções para tratamento de dados ruidosos advindos do mundo real, cuidando de se preocupar em trabalhar em conjunto com o RTOS Zephyr, economizando tempo de desenvolvimento e centralizando código aberto para todo e qualquer projetista de software interessado em consumir e contribuir ao projeto.


\section{Organização do Trabalho}
Este trabalho está organizado da seguinte forma, para o capítulo 1 apresentou-se a introdução da pesquisa, com dados e justificativas baseadas na bibliografia de suas áreas, contendo também os objetivos do trabalho, além do escopo e justificativas. Já no capítulo 2 apresenta-se uma revisão bibliométrica acerca dos trabalhos relacionados e do escopo deste projeto. O capítulo 3 contém aspectos da metodologia adotada e os requisitos necessários, com o  capítulo 4 
% TODO: Inserir mudanças para projeto novo
possuindo os resultados obtidos detalhadamente, que por último contendo o capítulo 5 traz as conclusões retiradas desta pesquisa, com suas possíveis contribuições e benefícios para a sociedade, comunidade científica e de desenvolvimento em sistemas embarcados.
 

% Texto corrido