%% abtex2-modelo-trabalho-academico.tex, v-1.9.2 laurocesar
%% Copyright 2012-2014 by abnTeX2 group at http://abntex2.googlecode.com/ 
%%
%% This work may be distributed and/or modified under the
%% conditions of the LaTeX Project Public License, either version 1.3
%% of this license or (at your option) any later version.
%% The latest version of this license is in
%%   http://www.latex-project.org/lppl.txt
%% and version 1.3 or later is part of all distributions of LaTeX
%% version 2005/12/01 or later.
%%
%% This work has the LPPL maintenance status `maintained'.
%% 
%% The Current Maintainer of this work is the abnTeX2 team, led
%% by Lauro César Araujo. Further information are available on 
%% http://abntex2.googleco@misc{MarileneCarneiroMatos,

%%
%% This work consists of the files abntex2-modelo-trabalho-academico.tex,
%% abntex2-modelo-include-comandos and abntex2-modelo-references.bib
%%
% ------------------------------------------------------------------------
% ------------------------------------------------------------------------
% abnTeX2: Modelo de Trabalho Academico (tese de doutorado, dissertacao de
% mestrado e trabalhos monograficos em geral) em conformidade com 
% ABNT NBR 14724:2011: Informacao e documentacao - Trabalhos academicos -
% Apresentacao
% ------------------------------------------------------------------------
% ------------------------------------------------------------------------

\documentclass[
	% -- opções da classe memoir --
	12pt,				% tamanho da fonte
	openright,			% capítulos começam em pág ímpar (insere página vazia caso preciso)
	oneside,			% para impressão em verso e anverso. Oposto a oneside
	a4paper,			% tamanho do papel. 
	% -- opções da classe abntex2 --
	%chapter=TITLE,		% títulos de capítulos convertidos em letras maiúsculas
	%section=TITLE,		% títulos de seções convertidos em letras maiúsculas
	%subsection=TITLE,	% títulos de subseções convertidos em letras maiúsculas
	%subsubsection=TITLE,% títulos de subsubseções convertidos em letras maiúsculas
	% -- opções do pacote babel --
	english,			% idioma adicional para hifenização
	brazil				% o último idioma é o principal do documento
	]{abntex2}

% ---
% Pacotes básicos 
% ---
\usepackage{unitins}
\usepackage{lmodern}			% Usa a fonte Latin Modern			
\usepackage[T1]{fontenc}		% Selecao de codigos de fonte.
\usepackage[utf8]{inputenc}		% Codificacao do documento (conversão automática dos acentos)
\usepackage{lastpage}			% Usado pela Ficha catalográfica
\usepackage{indentfirst}		% Indenta o primeiro parágrafo de cada seção.
\usepackage{color}				% Controle das cores

\usepackage{graphicx}			% Inclusão de gráficos
\usepackage{subfig}
\usepackage{microtype} 			% para melhorias de justificação
\usepackage{soul}
\usepackage{amssymb}
\usepackage{amsmath}
\usepackage{spreadtab}
\usepackage{multirow}
\usepackage{amsthm}
\usepackage{url}

%Adicionado para criar Matrizes, cases e layouts no latex
\usepackage{array}
\usepackage{mathdots}

\usepackage{float}

\usepackage[portuguese, ruled, linesnumbered]{algorithm2e}


\usepackage{algorithmic}

% Utilizado para customizar as fontes das equações 
\usepackage{fixmath}
\usepackage{verbatim}

% ---
		
% ---
% Pacotes adicionais, usados apenas no âmbito do Modelo Canônico do abnteX2
% ---
		% para geração de dummy text
% ---

% ---
% Pacotes de citações
% ---


%\usepackage[brazilian,hyperpageref]{backref}	 % Paginas com as citações na bibl
\usepackage[alf,abnt-etal-list=2,abnt-repeated-author-omit=yes]{abntex2cite}	% Citações padrão ABNT

\usepackage{multirow}
\usepackage[table,xcdraw]{xcolor}
\usepackage{colortbl}
\definecolor{lightgray}{gray}{0.9}
\graphicspath{{imagens/}}

%INCLUDE PDF 

\usepackage{pdfpages}

% LONG TABLE - TABELA LONGA 

%\usepackage[margin=1in]{geometry}
\usepackage{longtable,tabularx,ltxtable}

\usepackage{threeparttablex} % for "ThreePartTable" environment
\usepackage{pgfplotstable}
\usepackage{filecontents}
\usepackage{makecell}
\usepackage{booktabs}
\usepackage{float}

\usepackage[brazil]{babel} % for European Portuguese use portuguese
\usepackage[fixlanguage]{babelbib}
\selectbiblanguage{brazil}
\citeoption{abnt-etal-cite=2}


\usepackage[running]{lineno}

% \newcolumntype{Y}{>{\centering\arraybackslash}X}
%\setlength\LTleft{0pt}
%\setlength\LTright{0pt}

%\usepackage[justification=centering]{caption}

% \theoremstyle{theorem}
%\newtheorem{theorem}{Theorem}
%\newtheorem{teo}{Teorema}[chapter]
%\newtheorem{lema}[teo]{Lema}
%\theoremstyle{definition}
%\newtheorem{defi}[teo]{Definição}

% --- 
% CONFIGURAÇÕES DE PACOTES
% --- 

% ---
% Configurações do pacote backref
% Usado sem a opção hyperpageref de backref
%\renewcommand{\backrefpagesname}{Citado na(s) página(s):~}
% Texto padrão antes do número das páginas
%\renewcommand{\backref}{}
% Define os textos da citação
%\renewcommand*{\backrefalt}[4]{
%	\ifcase #1 %
%		Nenhuma citação no texto.%
%	\or
%		Citado na página #2.%
%	\else
%		Citado #1 vezes nas páginas #2.%
%	\fi}%
% ---

% ---
% Informações de dados para CAPA e FOLHA DE ROSTO
% ---
\titulo{Avaliação do Sistema Operacional de Tempo Real Zephyr na
Plataforma ESP32}
\autor{Tales Monteiro Melquiades}
\local{Palmas}
\data{2021}
\orientador{Prof. Me. Marco Antonio Firmino De Sousa}
\instituicao{%
  Universidade Estadual do Tocantins
  \par
  Curso de Sistemas de Informação
  }
\tipotrabalho{Projeto de conclusão de curso (Graduação)}
% O preambulo deve conter o tipo do trabalho, o objetivo, 
% o nome da instituição e a área de concentração 
\preambulo{Projeto apresentado como requisito para aprovação
na disciplina de Projeto de conclusão de Curso de
Sistemas de Informações da Universidade Estadual
do Tocantins - UNITINS, sob a Orientação do Prof.
Me. Marco Antonio Firmino De Sousa.}

% ---


% ---
% Configurações de aparência do PDF final

% alterando o aspecto da cor azul
\definecolor{blue}{RGB}{41,5,195}

% informações do PDF
\makeatletter
\hypersetup{
     	%pagebackref=true,
		pdftitle={\@title}, 
		pdfauthor={\@author},
    	pdfsubject={\imprimirpreambulo},
		%pdfcreator={LaTeX with abnTeX2},
		pdfkeywords={Algoritmo}{trabalho acadêmico}, 
		colorlinks=true,       		% false: boxed links; true: colored links
    	linkcolor=blue,          	% color of internal links
    	citecolor=blue,        		% color of links to bibliography
    	filecolor=magenta,      		% color of file links
		urlcolor=blue,
		bookmarksdepth=4
}
\makeatother
% --- 

% --- 
% Espaçamentos entre linhas e parágrafos 
% --- 

% O tamanho do parágrafo é dado por:
\setlength{\parindent}{1.3cm}

% Controle do espaçamento entre um parágrafo e outro:
\setlength{\parskip}{0.2cm}  % tente também \onelineskip

% ---
% compila o indice
% ---
\makeindex
% ---

% ----
% Início do documento
% ----
\begin{document}

% Retira espaço extra obsoleto entre as frases.
\frenchspacing

% ----------------------------------------------------------
% ELEMENTOS PRÉ-TEXTUAIS
% ----------------------------------------------------------
% \pretextual
% ----------------------------------------------------------

% ----------------------------------------------------------
% ELEMENTOS PRÉ-TEXTUAIS
% ----------------------------------------------------------
% \pretextual

% ---
% Capa
% ---
\imprimircapa
% ---

% ---
% Folha de rosto
% (o * indica que haverá a ficha bibliográfica)
% ---
\imprimirfolhaderosto
% ---

% ---
% Inserir folha de aprovação
% ---

% Isto é um exemplo de Folha de aprovação, elemento obrigatório da NBR
% 14724/2011 (seção 4.2.1.3). Você pode utilizar este modelo até a aprovação
% do trabalho. Após isso, substitua todo o conteúdo deste arquivo por uma
% imagem da página assinada pela banca com o comando abaixo:
%
% \includepdf{folhadeaprovacao_final.pdf}
%
\begin{folhadeaprovacao}



	\begin{center}
		\includegraphics[width=1\textwidth]{imagens/unitins.png}
		\ABNTEXchapterfont\Large   CURSO DE SISTEMAS DE INFORMA{\c{C}}{\~{A}}O

		\par
		\vspace*{1cm}
		{\ABNTEXchapterfont\bfseries\large \expandafter\MakeUppercase{\imprimirtitulo}  \vspace*{1cm}    }
		\par
		{\large \expandafter\MakeUppercase{\imprimirautor}}
		%\vspace*{\fill}
		\par
		\vspace*{1cm}
		\hspace{.45\textwidth}
		\begin{minipage}{.5\textwidth}
			\small\imprimirpreambulo

		\end{minipage}%
		%	\vspace*{\fill}
	\end{center}



	\assinatura{\textbf{\imprimirorientador} \\ Orientador}
	\assinatura{\textbf{Professor} \\ Convidado 1}
	\assinatura{\textbf{Professor} \\ Convidado 2}
	%\assinatura{\textbf{Professor} \\ Convidado 3}
	%\assinatura{\textbf{Professor} \\ Convidado 4}

	\begin{center}
		\vspace*{0.5cm}
		{\large\imprimirlocal}
		\par
		{\large\imprimirdata}
		\vspace*{1cm}
	\end{center}

\end{folhadeaprovacao}
% ---

% ---
% Dedicatória
% ---
\begin{dedicatoria}
	\vspace*{\fill}
	\centering
	\noindent
	\textit{A todas as pessoas que direta ou indiretamente contribuíram para a realização deste trabalho.} \vspace*{\fill}
\end{dedicatoria}
% ---

% ---
% Agradecimentos
% --- Es
\begin{agradecimentos}
	Ao meu orientador, pelo suporte no pouco tempo que lhe coube, pelas suas correções e incentivos.
	Agradeço a todos os  meus professores por me proporcionar o conhecimento e afetividade da educação 
	no processo de formação profissional, pela excelência da qualidade técnica de cada um.
	Aos meus pais que sempre me incentivaram a superar as dificuldades.
	Aos meus amigos de jornada, por não me deixarem desistir.
	A todos que direta ou indiretamente fizeram parte de minha formação, o meu muito obrigado.


\end{agradecimentos}
% ---

% ---
% Epígrafe
% ---
\begin{epigrafe}
	\vspace*{\fill}
	\begin{flushright}
		\textit{``The way to the stars is open.''\\
			Sergei Korolev}
	\end{flushright}
\end{epigrafe}
% ---

% ---
% RESUMOS
% ---

% resumo em português
\setlength{\absparsep}{18pt} % ajusta o espaçamento dos parágrafos do resumo
\begin{resumo}
	A leitura de dados vindos de sensores em tempo real sempre foi uma tarefa difícil e custosa. Há uma grande quantidade de ruído proveniente da baixa qualidade dos sensores e sequelas resultantes do ambiente, que produzem uma imensa quantidade de dados imprecisos e incorretos, estes os quais atrapalham na tomada de decisão de sistemas embarcados. 
	% O problema é muito estudado pela comunidade científica, diversos trabalhos apontam diferentes formas de tratar e ignorar ruídos advindos de sensores. Notou-se a grande presença de artigos recentes que aprofundam o assunto em diversas áreas do conhecimento onde a leitura de dados tem um papel crítico e decisivo no resultado final, os quais são tratados e apresentados aqui com uma revisão bibliométrica que apresenta uma visão dos termos, artigos e autores dos últimos 5 anos desde a concepção deste trabalho. 
	Neste trabalho foi implementado uma função de intervalo de confiança móvel, que pode ser utilizada em conjunto com outros métodos para obter dados mais precisos sem interferir no resultado final, o código tem como objetivo trabalhar de forma amigável em ambiente de sistema embarcado onde memória e processamento são recursos escassos e valiosos. Os resultados mostraram que o intervalo de confiança pode ser utilizado em certas ocasiões onde o tempo de espera não é tão custoso, retornando dados sempre dentro de um padrão aceitável de leitura mas perdendo qualidade na leitura de grandes curvas.
	
	% Neste trabalho também foi implementado funções de filtro populares e uma função de filtragem probabilística, em um repositório de código aberto apelidado de zscilib, focado em um conjunto de funções úteis para computação científica, análise de dados e manipulação de dados no contexto de dispositivos de hardware embarcado, desenvolvido especialmente para o sistema operacional de tempo real Zephyr mantido pela fundação Linux. As contribuições aqui propostas serão integradas a comunidade de código aberto e servirão para que outros pesquisadores, estudantes e atuantes da área possam usufruir e contribuir com a implementação código proposto.



	\textbf{Palavras-chaves}: Algoritmo de filtragem, Sistemas embarcados, Software, Sensor, Redução de ruído.
\end{resumo}

% resumo em inglês
\begin{resumo}[Abstract]
	\begin{otherlanguage*}{english}
		...
		\vspace{\onelineskip}

		\noindent
		\textbf{Key-words}: Filtering algorithm, Embedded systems, Software, Sensor, Noise reduction.
	\end{otherlanguage*}
\end{resumo}


% ---
% inserir lista de ilustrações
% ---
\pdfbookmark[0]{\listfigurename}{lof}
\listoffigures*
% ---

% ---
% inserir lista de tabelas
% ---
\newpage
\pdfbookmark[0]{\listtablename}{lot}
\newpage
\listoftables*
\cleardoublepage
% ---

% ---
% inserir lista de abreviaturas e siglas
% ---
\begin{siglas}
	\item RTOS - Sistema operacional de tempo-real.
	\item ISR - Rotinas de serviço de interrupção.
	\item SO - Sistema operacional.
	\item IC - Intervalo de confiança.
	\item SRAM - Memória Estática de Acesso Aleatório
	\item ROM - Memória de somente leitura
\end{siglas}
% ---


% ---
% inserir o sumario
% ---
\pdfbookmark[0]{\contentsname}{toc}
\tableofcontents*
\cleardoublepage
% ---



% ELEMENTOS TEXTUAIS
% ---------------------------------[!htb]-------------------------
%\textual

%Capitulos
% ----------------------------------------------------------
% Introdução (exemplo de capítulo sem numeração, mas presente no Sumário)
% ----------------------------------------------------------

\chapter{Introdução}\label{intro}
\section{Conceitos Introdutórios e Problematização}
Com os avanços recentes na miniaturização de componentes eletrônicos de prateleira (COTS), construir 
satélites deixou de ser exclusividade de governos e grandes empresas apoiadas por governos, agora 
universidades, empresas menores e grupos entusiastas podem construir e lançar ao espaço em um ou dois 
anos pequenas espaçonaves de baixo custo, baixa potência com componentes de prateleiras prontos para 
uso \cite{Poghosyan2017}.

Está tecnologia vem permitindo que países em desenvolvimento e emergentes possam entrar no setor espacial, 
o desenvolvimento, operação e análise de dados dessas pequenas espaçonaves promovem e estimulam a educação 
científica em mais de 80 universidades em todo o mundo que possuem programas nessas áreas, estes programas 
ocasionam uma importante atividade comercial, com muitos desses projetos surgindo como projetos acadêmicos 
\cite{Woellert2011}. O padrão de um CubeSat de uma unidade (1U) e de 10 x 10 x 10 cm³, tendo seu peso 
limitado a 1,33 kg podendo ser composto por um ou mais unidades seguindo o mesmo padrão, este padrão foi 
inicialmente desenvolvido em 1999 pelos professores Jordi Puig-Suari na Cal Poly e Bob Twiggs na Universidade 
de Stanford, cada uma dessas espaçonaves pode ser lançadas ao custo de 50 a 200 mil dólares\cite{Selva2012}, 
este custo pode variar dependendo das tecnologias empregadas e plataforma de lançamento.

% \begin{figure}[H]
% 	\centering
% 	\includegraphics[width=15cm]{imagens/cubesat_nasa_image.jpg}
% 	\caption{CubeSat RadFxSat}
% 	Fonte: Portal "www.nasa.gov", The Radio Amateur Satellite Corporation (AMSAT) e Vanderbilt University.
% 	\label{fig: CubeSat RadFxSat}
% \end{figure}

A fim de diminuir custos estas plataformas podem ser construídas com tecnologias de código aberto. Os CubeSat's 
de código aberto populares possuem uma característica em comum, utilizam o sistema operacional de tempo real de 
código aberto FreeRTOS, pois além de ser um dos RTOS (Sistema operacional de Tempo Real) mais adotado para projetos e um dos mais populares segundo 
a \cite{Lynx}. E um dos RTOS com maior documentação e maior comunidade na internet, sendo um dos favoritos em 
projetos de código aberto por ter maior aceitação e entendimento do publico. O mesmo e pequeno, simples e confiável 
sendo amplamente usado na indústria de microeletrônica, sendo escrito em C com partes em assembler \cite{nicolas2019avaliaccao}.


Desta forma, pretendeu-se avaliar um novo sistema operacional de tempo real para a utilização do mesmo no 
desenvolvimento de CubeSats, visto que o projeto Zephyr. De acordo com \cite{nyffenegger2020connecting}destaca 
que o Zephyr RTOS é um sistema relativamente novo no mercado, destacando-se em sua construção pensada para operar 
em ambientes restritos a poucos recursos de memoria e consumo de energia, permitindo que aplicativos complexos 
sejam executados em dispositivos altamente limitados. É altamente modular, permitindo que os usuários configurem 
o funcionamento do sistema de acordo com as necessidades do projeto sem prejudicar os requisitos, entretanto 
por se tratar de um sistema novo, implica em ter um suporte difícil para iniciantes, exigindo um nível de 
habilidade alta na compreensão e configuração de sistemas Linux com Makefiles, um ponto destacado negativamente 
está na carência de resoluções de problemas na internet e documentação para iniciantes fraca de explicações, 
tornando a curva de aprendizado bastante íngreme. Utilizar Zephyr em um projeto de Cubesat não 
e uma ideia muito distante, abrindo uma janela de possíveis trabalhos na área que possam 
trazer grandes contribuições, possibilitando o surgimento deste trabalho.


Em \cite{Slacka} o autor destaca algumas dificuldades ao se desenvolver um sistema operacional para missões de 
cubesat, pois nem sempre o sistema poderá estar acessível a um operador humano, tornando a responsabilidade do 
software em tomar atitudes em tempo real mais desafiadoras, o sistema precisa ter confiabilidade e tolerância a 
falhas de software e hardware. Também informa que sistemas operacionais de tempo real como o 
FreeRTOS empregam muitas funcionalidades complexas e custosas que não necessariamente serão empregadas em uma 
missão de cubesat.


\section{Objetivos e Escopo de Pesquisa}
\subsection{Objetivos de Pesquisa}
Visto a exorbitante quantidade de dados provindo de sensores, somando a incerteza da qualidade de fabricação dos componentes eletrônicos de baixo custo disponíveis em ampla quantidade no mercado, torna o desenvolvimento para sistemas criticos que dependem da veracidade dos valores provindos dos sensores, ambiente assíncrono com múltiplas tarefas em simultâneo, complexos e custosos para as equipes de engenheiros de software, que utilizam de tempo para garantir a integridade e veracidades de todos os valores. 

Com isso em mente o trabalho aqui proposto se dispõem de construir uma biblioteca de código aberto, que disponibilizara funções adequadas para a filtragem de dados provindo de sensores em ambiente de tempo real com multitarefas, testando a coleta e filtragem de dados no processador dual core ESP32 no sistema operacional de tempo real \cite{Zephyr}.

\subsection{Objetivos secundários}
Os objetivos secundários a serem alcançados são:
\begin{itemize}
\item Realizar uma revisão bibliométrica de trabalhos que utilizam de métodos probabilísticos para filtragem de dados de sensores
\item Disponibilizar o código de forma aberta a comunidade, para que possa ser utilizado por qualquer outro interessado em tratar dados de sensores em ambiente de multitarefas
\item Escrever funções de filtragem utilizando os métodos de Desvio padrão e Intervalo de confiança
\end{itemize}


\subsection{Escopo da Pesquisa}
O escopo desta pesquisa abrange um conjunto de funções escritas na linguagem C para a filtragem de dados indesejáveis vindos de sensores diversos, funções quais... % Inserir métodos e pesquisas relevantes.
, tento como característica atuar de forma amigável junto a um sistema operacional de tempo real. Assim, em um primeiro momento considerou-se desenvolver e testar a biblioteca sobre o Chip ESP32 da fabricante Espressif e em conjunto com o sistema operacional Zephyr um projeto mantido pela Fundação Linux.  


\section{Justificativas}
A comunidade de desenvolvimento de CubeSats educacionais de código aberto utilizam em sua grande maioria de 
poucos RTOS disponíveis no mercado, exemplo o FreeRTOS, em virtude de sua já confirmada performance e garantia 
em embarcada, considerando os poucos recursos para empregar um sistema ainda não validado na arquitetura do 
projeto.
Visto que o ESP32 é um chip muito popular e já validado em viagens espaciais, destaca-se que 
a contribuição para o setor, por meio desta pesquisa, 


% Validar isso
será a avaliação de um sistema novo e que promete ser mais 
eficiente em fatores como memoria, espaço e energia, sendo que o mesmo ainda não foi validado em uma missão 
real, 



assim então economizando tempo e dinheiro das equipes e universidades que usufruem de projetos de CubeSat.


\section{Organização do Trabalho}
% Inserir resultados
Este trabalho está organizado da seguinte forma, para o capítulo 1 apresentou-se a introdução da 
pesquisa, com dados e justificativas baseadas na bibliografia de suas áreas, contendo 
também os objetivos do trabalho, além do escopo e justificativas. Já no capítulo 2 apresenta-se 
uma revisão bibliométrica acerca dos trabalhos relacionados e do escopo deste projeto. O 
capítulo 3 contém aspectos da metodologia adotada e os requisitos necessários, com o capítulo 
4 possuindo a definição do experimento e de seus criterios detalhadamente, que por ultimo 
contendo o capítulo 5 traz as conclusões retiradas desta pesquisa, com suas possiveis contribuições 
e beneficios para a sociedade, comunidade científica e de desenvolvimento de CubeSats.

% Texto corrido
% ---
% Capitulo de revisão de literatura
% ---



\chapter{Revisão Bibliométrica sobre Filtragem de dados de sensores}\label{referencial_teorico}

Nesta seção são consideradas as informações referentes à produção acadêmica mundial, que abordam os termos em inglês Noise reduction, noise abatement, Filtering algorithm e Sensor, significando em sequência redução de ruído, algoritmos de filtragem e sensores.

Ambos os tópicos foram pesquisados na base IEEE Xplore, com a busca limitada aos termos noise reduction e noise abatement no título do documento e filtering algorithm, sensor e noise reduction apenas no texto completo, delimitando exclusivamente aos artigos de 2018 a 2022. Resultando em 675 resultados sendo eles 450 conferências, 215 artigos, 8 artigos com acesso  antecipado e 2 revistas, foi realizado uma revisão bibliométrica sobre estes 215 artigos resultantes.

Dados oriundos de sensores são constantemente bombardeados com interferências aleatórias do ambiente onde se encontram, também devido a baixa qualidade provenientes dos sensores de baixo custo disponíveis, inserindo valores incorretos que são caracterizados como ruídos nas amostras. Esse problema desperta um grande interesse de pesquisadores, que buscam lidar com o tratamento e a filtragem de dados ruidosos provindos de sensores diversos, o artigo \cite{chiang_noise_reduction_in_ECG} se destaca por ser o trabalho mais citado nessa pesquisa bibliométrica, o mesmo aborda o problema da interferência de ruídos nos sinais provindos dos sensores de eletrocardiograma, esses dados podem ser contaminados por fatores como a estática da pele ou mesmo pela respiração do paciente, para isso técnicas como wavelet são muito populares, utilizadas para filtrar dados cancelando o ruído analisando mudanças bruscas ou picos na frequência do sinal, o método de decomposição do modo empírico também é bastante utilizado, definindo fronteiras entre o local máximo e mínimo de uma subtração de sequência que consequentemente ajudam na triagem do tratamento do sinal.

\begin{figure}[H]
	\centering
	\includegraphics[width=15cm]{anexos/ris/IEEE/Noise_reduction_and_noise_abatement_andsensor_filtering_algorithm/network_visualization_with_lines.png}
	\caption{Mapa de visualização de rede}
	Fonte: Autor com base no Software VOSViwer.
	\label{fig: network_visualization_with_lines}
\end{figure}

A Figura~\ref{fig: network_visualization_with_lines} representa um mapa de visualização de rede, onde pode-se visualizar os termos mais predominantes. Aqui são percebidos o termos que se repetiram mais de 5 vezes nos textos e resumos, nota-se a formação de grupos de acordo com suas áreas de atuação, alguns termos em destaque são em azul relacionados a motores, vibrações e forças mecânicas, os em verdes referentes a componentes eletrônicos, roxo a sensores ópticos, amarelo o tratamento de imagem e em vermelho sensores embarcados e processamento de sinais. 


Todas essas expressões aprofundam a problemática no qual tratamento de ruídos, algoritmos de filtragem e sensores estão correlacionados, indo de problemas em programas de computador a desenvolvimento de circuitos eletrônicos e peças mecânicas. O erro na fabricação de sensores ópticos pode levar a fenômenos de interferência nos resultados \cite{liu_interference_stripe}, essas anomalias prejudicam sistemas de medição holográfica digital, a fim de melhorar a captação de imagens sem ruído o autor propõem um novo método de processamento de imagem utilizando pirâmide laplaciana para destacar o ruído de faixa de interferência.

\begin{figure}[H]
	\centering
	\includegraphics[width=15cm]{anexos/ris/IEEE/Noise_reduction_and_noise_abatement_andsensor_filtering_algorithm/overlay_visualization_cites.png}
	\caption{Mapa de visualização autores}
	Fonte: Autor com base no Software VOSViwer.
	\label{fig: overlay_visualization_cites}
\end{figure}

A Figura~\ref{fig: overlay_visualization_cites} apresenta um mapa semelhante a figura anterior, a imagem destaca os autores mais citados em média, destacando-os pelo tamanho do círculo atrás de seu nome, e pela sua presença em artigos mais recentes de acordo com a cor mais amarelada.
Em \cite{duarte_speckle_noise} afim de melhorar aplicações biomédicas em imagens de ultrassom, o trabalho descreveu 27 técnicas para tratamento eliminação de ruído para ultrassom, essas imagens são de verdadeira importância para o diagnóstico clínico e procedimentos terapêuticos não invasivos, fundamentais em diversas áreas da saúde.

Na Figura~\ref{fig: overlay_visualization} abaixo visualiza-se um mapa de sobreposição, onde os vocabulários mais amarelos representam os que se encontram em publicações em média mais recentes, também podemos notar a relação entre as palavras encontradas em todos os títulos e resumos.

\begin{figure}[H]
	\centering
	\includegraphics[width=15cm]{anexos/ris/IEEE/Noise_reduction_and_noise_abatement_andsensor_filtering_algorithm/overlay_visualization.png}
	\caption{Mapa de sobreposição}
	Fonte: Autor com base no Software VOSViwer.
	\label{fig: overlay_visualization}
\end{figure}

Há de se notar que os termos filtragem e algoritmos de filtragem se destacam pela sua quantidade acima da média de trabalhos mais recentes, não se distanciando das definições de redução de ruído, sensores e algoritmos de processamento de sinais próximos do centro do mapa.
 


\subsection{Conclusões sobre a Revisão Bibliométrica realizada}
Percebe-se que o estudo na área de tratamento de dados de sensores concentra em média uma grande quantidade de trabalhos recentes, com diferentes abordagens de como remover ou reduzir dados ruidosos, apresentando a ideia de que o tema é de interesse atual da comunidade acadêmica mundial. A diversos ramos nos quais tratamento e eliminação de ruídos de sensores podem peregrinar, podendo verificar a ocorrência dos termos em problemas que não necessariamente estão interessados na obtenção do sinal limpo, mas sim na caracterização e coleta dos ruídos na amostra ou que fogem do escopo deste trabalho com a eliminação de ruído, sendo feita através de equipamento físico.



\section{Trabalhos Relacionados}%\label{referencial_teorico}
% Aqui será dedicado a descrever os trabalhos relacionados ao tema desta pesquisa.

O trabalho \cite{kalambet2011noise} descreve um método de filtragem de ruído baseado em intervalo de confiança, com uma abordagem que utiliza matrizes para evitar dados discrepantes deixando-os como estão, o estudo enfatiza alguns dos problemas encontrados nos filtros mais utilizados, como a falta de critério claro da filtragem e interferência no resultado final. A pesquisa \cite{madhale2020adaptive} apresenta uma técnica de remoção de ruído adaptável baseada em intervalo de confiança adaptável e orientado a dados, para limpeza de valores do monitoramento através de eletroencefalograma que analisa a atividade elétrica cerebral espontânea, o trabalho obteve sucesso reduzindo significamente as taxas de ruído tornando a dinâmica do cérebro mais analisável no exame e com menos interferência.

\subsection{Tabela de referencias}
Aqui serão apresentados os 5 artigos recentes relacionados a esse trabalho, tendo suas vantagens e desvantagens levantadas com relação aos seus métodos propostos e uma breve descrição do trabalho.

\begin{longtable}{|p{2cm}|p{4cm}|p{3.5cm}|p{3.5cm}|}
    \hiderowcolors
    \caption{Referências bibliográficas}
    \label{tab:makespan}\\
    \showrowcolors
    \hline
    \rowcolor[HTML]{C0C0C0} 
    \multicolumn{1}{c|}{\cellcolor[HTML]{C0C0C0}\textbf{Trabalhos analisados}} & \multicolumn{1}{c|}{\cellcolor[HTML]{C0C0C0}\textbf{Objetivo}} & \multicolumn{1}{c|}{\cellcolor[HTML]{C0C0C0}\textbf{Vantagens}} & \multicolumn{1}{c|}{\cellcolor[HTML]{C0C0C0}\textbf{Desvantagens}} \\ \hline

    \endfirsthead
    \rowcolor[HTML]{C0C0C0} 
    \multicolumn{1}{c|}{\cellcolor[HTML]{C0C0C0}\textbf{Trabalhos analisados}} & \multicolumn{1}{c|}{\cellcolor[HTML]{C0C0C0}\textbf{Objetivo}} & \multicolumn{1}{c|}{\cellcolor[HTML]{C0C0C0}\textbf{Vantagens}} & \multicolumn{1}{c|}{\cellcolor[HTML]{C0C0C0}\textbf{Desvantagens}} \\ \hline

    \endhead
    \hline
    \cite{Arab_LSTM_ResNet} &   O trabalho propõe utilizar uma técnica de aprendizado profundo para classificar e eliminar ruídos de sistemas de comunicação via micro-ondas, aproveitando-se da aptidão do algoritmo em se adestrar se com os dados coletados em tempo real.	& Algoritmo pode ser treinado em tempo real e acompanhar diferentes tipos de ruídos. & Necessita de uma grande quantidade de dados já coletados para treinamento, exigência de grande capacidade de recurso de computação. \\ \hline
   
    \cite{Kamata_mems} &   O seguinte estudo propõe uma filtragem para processamento de sinal de um componente eletrônico giroscópio embarcado e avaliando seu desempenho. &   A técnica pode ser utilizada também para acelerômetros, e viabiliza o uso de componentes de custo baixo e alta precisão. & Limitasse a um ambiente de sensores específicos. \\ \hline
    
    \cite{Ning_magnetometer} &   Aqui os autores implementam uma combinação de filtros para eliminar dados ruidosos em tempo real e compensar a interferência dos erros de um sensor magnético, utilizando de diversos métodos como Auto-Regressão e Média móvel, para modelar a medição de ruído, a fim de excluir dados errados do resultado final. &   O método utilizado é adaptativo e abrangente, podendo resolver ruídos dinamicamente. & Os testes não foram realizados durante o processo de coleta de dados. \\ \hline
    
    \cite{Kaan_emg} &   Aqui é proposto um novo método de processamento de dados de sensores de eletromiografia sensível, utilizando um algoritmo adaptativo em tempo real para eliminar os ruídos provindos de fontes elétricas de corrente alternada, sem perturbar os dados reais do sensor, ao qual conseguiu superar cinco alternativas existentes de última geração para tratamento de sinal de eletromiografia, mantendo a qualidade do sinal.  &  Seu comportamento adaptativo exibe uma vantagem em manter os dados coletados o mais próximo possível dos dados reais, sem diminuir a potência do sinal. & Por estar no estado da arte, ainda não apresenta outros estudos comprovando sua utilização em tempo real. \\ \hline
    
    \cite{Zhou_ambient} &   Os autores apresentam um método que combina processamento de sinal e cancelamento de ruído adaptativo para filtrar e eliminar interferências do ambiente. Duplicando o sinal recebido em dois canais diferentes, sobrepondo um sobre o outro e eliminando as interferências em ambos de forma a evitar perdas dos espectros. &   Identifica sinais de interferência utilizando uma quantidade inferior de dados. & Quanto maior a sobreposição, maior a carga computacional necessária para processar os canais diferentes. \\ \hline
    
\end{longtable}
 

% Ademais, 
% o levantamento bibliométrico passa a ideia que utilizar Zephyr em um projeto de Cubesat não 
% e uma ideia muito distante, abrindo uma janela de possíveis trabalhos na área que possam 
% trazer grandes contribuições, possibilitando o surgimento deste trabalho.

% Conclusão


% \subsection{Problema da interferência em sensores}

\section{Intervalo de confiança}
Segundo \cite{patino2015intervalos} um IC é a metade da divisão do tamanho do real efeito na população de interesse, essa imprecisão ou diferença entre duas médias é sempre a melhor estimativa dado o tamanho da população atingida. De acordo com \cite{henriques2011dificuldades} o intervalo de confiança é um dos procedimentos gerais de inferência estatística que pode aplicar-se a diversos campos de problemas. Os mais utilizados são estimação de parâmetros desconhecidos de uma população, comparação de distribuições, teste de hipóteses sobre parâmetros populacionais, determinar o tamanho de amostra adequado para realizar uma inferência, e determinar limites de tolerância. Cada um destes campos é muito alargado e variado e incluem a estimação de média, proporção, variâncias, parâmetros de regressão e correlação.  


\section{ Filtro de Kalman}
\cite{tan2005sensoclean} diz que através de uma sequência de dados observados o filtro de Kalman pode estimar o estado verdadeiro de um sistema dinâmico, o mesmo é utilizado por uma grande parte das aplicações de engenharia que visam sistemas complexos como visão computacional ou radares. O filtro é baseado em probabilidade estatística sendo capaz de suavizar ruídos de sensores eletrônicos visando seu posterior processamento \cite{International_Conference__Zhuang}.
 

\section{Média Móvel}
De acordo com \cite{santos2021educaccao}, a média móvel é uma técnica que consiste em calcular a média aritmética das observações mais recentes de uma série de dados temporais. Portanto, temos uma estimativa que não leva em consideração a observação mais antiga. Os autores também esclareceram que o nome média móvel foi usado porque a cada período as observações mais antigas são substituídas pelas mais recentes, então uma nova média é calculada.


\section{Média Móvel Ponderada}
Segundo \cite{ribeiro2020analise}, a média móvel ponderada é superior à média móvel simples porque realiza uma mudança de impacto entre os dados de demanda mais antigos e os mais recentes, o que pode revelar algumas tendências. Eles também afirmam que a média móvel simples atribui o mesmo peso a cada componente da série de dados, enquanto a média móvel ponderada permite atribuir um fator de ponderação a cada elemento onde a soma de todos os pesos é igual a um.



% \section{Filtro FIR}

% \section{Filtro IIR}

% ---
% Capitulo de METODOLOGIA
% ---


\chapter{Metodologia}\label{cap:metodologia}
Nesta seção apresenta-se os materiais e métodos utilizados no desenvolvimento do trabalho proposto.
\section{Materiais e Métodos}

O presente estudo classifica-se como uma pesquisa experimental, a pesquisa experimental segundo \cite{wazlawick2017metodologia} condiciona o pesquisador a lidar com diversas variáveis experimentais e variáveis observacionais visando levar possivelmente, correlações e dependências entre as elas, utilizando de técnicas de amostragem e testes de hipóteses. O mesmo diz que trabalhos desenvolvidos em cima de abordagens padronizadas e aceitas internacionalmente, apresentando dados empíricos é relevantes, se encaixam no nível mais maduro de pesquisa, onde o autor deverá apresentar os resultados usando métricas aceitas pela comunidade, através de observações e medições, implicando que o pesquisador provocará alterações sistemáticas no ambiente do experimento para se observar os resultados após cada intervenção produzida.

\subsection{Materiais}

Para a implementação do método de intervalo de confiança, além da realização dos testes com dados em tempo real, utilizou-se dos seguintes materiais citados a seguir na tabela.

\begin{longtable}{|p{4cm}|p{3.5cm}|}
    \hiderowcolors
    \caption{Equipamentos utilizados}
    \label{tab:makespan}\\
    \showrowcolors
    \hline
    \rowcolor[HTML]{C0C0C0} 
    \multicolumn{1}{c|}{\cellcolor[HTML]{C0C0C0}\textbf{EQUIPAMENTO}} & \multicolumn{1}{c|}{\cellcolor[HTML]{C0C0C0}\textbf{UNIDADE}} \\ \hline

    \endfirsthead
    \rowcolor[HTML]{C0C0C0} 
    \multicolumn{1}{c|}{\cellcolor[HTML]{C0C0C0}\textbf{EQUIPAMENTO}} & \multicolumn{1}{c|}{\cellcolor[HTML]{C0C0C0}\textbf{UNIDADE}} \\ \hline

    \endhead
		\hline
		\textcolor[rgb]{0.125,0.129,0.141}{ESP32-DevKitC v1 ESP-WROOM-32U} & 1                \\
		\hline
		\textcolor[rgb]{0.059,0.067,0.067}{Cabo USB-Micro USB}          & 1                \\
		\hline
		Protoboard 400 pontos                                           & 1                \\
		\hline
		Sensor de temperatura termistor de 100k                         & 1                \\
		\hline
    
    \end{longtable}

O equipamento utilizado trata-se de um kit de desenvolvimento, apelidado de ESP32 DEV Kit v1 que acompanha o controlador ESP-WROOM-32, microprocessador Tensilica Xtensa 32-bit LX6 de dois cores, clock de até 240MHz, memória ROM de 448KB e SRAM de 520KB, e um termistor genérico de 100k que varia sua resistência dependendo da temperatura do ambiente retornando os dados para a análise.


\subsection{Análise exploratória dos dados}
Para avaliar o método de IC, foi necessário criar um conjunto de dados coletados através de um sensor de temperatura termistor de 100k acomodado no pino 21 do kit de desenvolvimento ESP32, aqui serão descritas as características mais importantes do dataset.

O conjunto contém 1592 linhas sem valores ausentes e 544 valores únicos com os valores \ang{32.14}c e \ang{33.47}c sendo os que mais se repetem com 35 aparições, todos os valores flutuando entre \ang{1.05}c e \ang{97.35}c graus, tendo média de \ang{37.19}c, desvio padrão de 13.33, primeiro quartil de 31.80, segundo quartil 33.00 e terceiro quartil em 35.33.
Os valores foram coletados no intervalo de aproximadamente 2 minutos, com todos os valores sendo processados em python com as bibliotecas pandas, numpy, matplotlib e scipy.


Na Figura~\ref{fig: hist} abaixo visualiza-se uma concentração maior de valores entre \ang{20}c e \ang{40}c graus com os valores seguintes podendo estar entre ruídos e picos de temperatura capturados pelo sensor. 

\begin{figure}[H]
	\centering
	\includegraphics[width=15cm]{imagens/sensores/hist.png}
	\caption{histograma do conjunto de dados}
	Fonte: Autor com base na biblioteca pandas.
	\label{fig: hist}
\end{figure}

Tendo em vista que a média de temperatura do local de coleta da amostra estava entre \ang{33}c graus, pode-se visualizar uma grande quantidade de valores aproximados na Figura~\ref{fig: hist2}. 

\begin{figure}[H]
	\centering
	\includegraphics[width=15cm]{imagens/sensores/hist2.png}
	\caption{histograma do conjunto de dados entre 20 e 40 graus}
	Fonte: Autor com base na biblioteca pandas.
	\label{fig: hist2}
\end{figure}

\begin{figure}[H]
	\centering
	\includegraphics[width=15cm]{imagens/sensores/boxplot.png}
	\caption{Boxplot do conjunto de dados}
	Fonte: Autor com base na biblioteca matplotlib.
	\label{fig: boxplot}
\end{figure}

Verifica-se na Figuras~\ref{fig: boxplot} a grande presença de valores discrepantes acima de \ang{40}c e abaixo de \ang{20}c.

Abaixo na Figura~\ref{fig: bruto} pode-se visualizar todo o conjunto de dados, nota-se a presença de grandes picos de temperatura que foram induzidos na amostra propositalmente a fim de diferenciar o ruído do sensor de um pico de temperatura real, esses picos foram introduzidos através do aquecimento do sensor de temperatura duas vezes durante a coleta de dados.


\begin{figure}[H]
	\centering
	\includegraphics[width=15cm]{imagens/sensores/bruto.png}
	\caption{Plotagem gráfica do conjunto de dados completo}
	Fonte: Autor.
	\label{fig: bruto}
\end{figure}

As linhas verticais discrepantes indicam possíveis ruídos, onde a temperatura alcançou valores muito diferentes em períodos extremamente curtos. Esses ruídos podem ter sido gerados por baixa qualidade do termistor ou pela exposição ao ambiente dos fios de conexão entre o termistor e o microcontrolador.


\begin{figure}[H]
	\centering
	\includegraphics[width=15cm]{imagens/sensores/bruto_100_primeiras}
	\caption{Plotagem gráfica das 100 primeiras linhas do conjunto}
	Fonte: Autor.
	\label{fig: bruto_100p}
\end{figure}

Na Figura~\ref{fig: bruto_100p} percebe-se que a leitura do sensor sempre apresenta alguma variação, por mais pequena que seja, essa pequena diferença na coleta de dados pode não significar um problema, podendo ser tratada através de um simples cálculo de média. A criticidade dessa variação deve ser avaliada em cada projeto, com os dados ruidosos muito fora da média e esporádicos, interferindo significativamente no resultado final.

\begin{figure}[H]
	\centering
	\includegraphics[width=15cm]{imagens/sensores/bruto_100_ultimas.png}
	\caption{Plotagem gráfica das 100 ultimas linhas do conjunto}
	Fonte: Autor.
	\label{fig: bruto_100u}
\end{figure}

Por fim na Figura~\ref{fig: bruto_100u} podemos analisar os últimos 100 dados, nos quais nota-se a presença de um pico de temperatura que começa a subir em \ang{33.05}c com sua máxima perto dos \ang{36.0}c, essa característica pode ser importante por mais que esteja em um curto período de tempo, e pode ser distinguida de um ruído pela sua característica de subida e descida prolongada. 




% ---
% Capitulo de METODOLOGIA
% ---

\chapter{Definição de experimentos}\label{cap:definicoes}
\section{Introdução}
O experimento proposto neste trabalho seguirá da seguinte maneira, os testes serão
escritos na linguagem C nos dois RTOS escolhidos, ambos usarão funções iguais para
medida de tempo, cada teste terá seu arquivo individual para que não sofra
interferência de outros testes, todos serão disponibilizados na internet como código
aberto, afim que qualquer pessoa possa testar e validar os resultados aqui obtidos.

È muito importante que um sistema operacional de tempo real não deixe que tarefas
ocupem muito tempo para tomarem suas responsabilidades, isto impacta diretamente no
ciclo de toda a missão, tendo em vista que todo tempo e precioso para uma aplicação
que reaja quase em tempo real a qualquer situação. A medida de tempo será a subtração
do tempo de coleta inicial com o final, estas métricas são destinadas a oferecer
informações de tempo já que se supõe que o mesmo e muito importante para missões que
utilizem um RTOS.
% Definir o experimento
% Como vai ser feito
% Resultado esperado

\subsection{Velocidade de troca entre tarefas}
Durante o processo de encerramento de uma tarefa, o sistema operacional pode realizar
algumas operações, como vasculhar interrupções e quais as possíveis tarefas que
entrarão em seguida, tudo isso consome um tempo precioso para a aplicação
interferindo no tempo de troca para outra atividade, com o intuito de medir esse
tempo em milissegundos usaremos a função millis() para obter o tempo de troca entre
duas tarefas, subtraindo o tempo obtido no final da primeira tarefa e no começo da
segunda. Serão criadas duas tarefas,


%  Qual a politica de fila adotada?
uma com prioridade maior que a outra,


sendo a
com menor prioridade tendo a ação de despertar a outra, o tempo que leva para a outra
tarefa ser despertada será medido. Espera-se que a troca entre as tarefas seja a mais
rápida possível, não apresentando gargalos, e que a tarefa com menor prioridade consiga
despertar a outra tarefa, mesmo que seja de prioridade maior.

\subsection{Tempo de passagem de mensagens entre tarefas}
É muito importante que as tarefas possam se comunicar entre si, de preferência de
forma mais rápida possível evitando que se encontrem bloqueadas esperando alguma
resposta, o tempo de envio e recebimento será medido tendo entre tarefas com prioridades
iguais, quanto entre tarefas de prioridades diferentes, espera-se que o sistema evite
bloqueios após a solicitação de envio em uma mensagens. Durante uma missão as tarefas
trocaram muitas mensagens, já que muito da operação depende que os módulos do satélite
troquem informações entre si, a troca destas mensagens não pode prejudicar o andamento
de todo o sistema.

\subsection{Tempo de bloqueio e liberação de desbloqueio de Semáforo e Mutex}
Iremos medir o tempo que o RTOS leva para bloquear um Semáforo e Mutex, junto também
o tempo para liberação, este teste necessita apenas de uma tarefa, a qual solicita e
libera em seguida o semáforo ou mutex. È importante que este tempo seja o mais curto
possível, já que outras tarefas também podem depender desta mesma variável, sendo o
tempo de liberação mais importante. Durante uma missão varias

% Que tipo de operações serão realizadas?
tarefas

pediram permissão
para usar o mesmo recurso, sendo que o controle de acesso para evitar possíveis
conflitos responsabilidade dos Semáforos e Mutexes, pouco atraso na troca destes pode
ocasionar em possíveis problemas de prioridades, levando com que tarefas com prioridades
maiores levem mais tempo para conseguirem liberação.

\subsection{Tempo de resposta a eventos externos atraves de interrupção}
Durante uma missão o sistema lidara com diversas interrupções, sendo elas de alta
prioridade e devendo serem tratadas imediatamente, este teste esta muito
correlacionado com o primeiro, já que iremos medir o tempo que leva para uma tarefa
despertar através desta vez por uma

% Como será implementado essa rotina?
rotina de interrupção.

O controlador da missão
terá que lidar com diversas ISR durante a missão, sendo importante que nenhuma delas
fique para trás ou que atrapalhem o ciclo de vida do sistema.

\subsection{Velocidade de alocação de memoria}
Alocar memória para realizar alguma operação pode levar algum tempo, todo RTOS tem
suas funções para alocação de memória, e cabe a este teste coletar medida de tempo
nesta operação, seja qual for o tamanho de memória requisitada o sistema deve
entrega-la rapidamente, aqui selecionaremos uma região de memória RAM e o tempo para
aquisição e liberação da mesma será medida. Aqui se destaca cada RTOS com suas regras
de alocação de blocos de memória, poderemos notar qual a diferença das abordagens e se
adequam nas operações rápidas de uma tarefa.

\subsection{Velocidade de leitura e gravação em memoria flash (Cartão SD)}
Este teste é muito semelhante ao anterior, mas se difere por utilizar de memória flash,
aqui mais especifico duas memórias, já que faremos testes alocando memoria flash
internado ESP32 como também um cartão SD externo. Em muitas missões de CubeSat memoria
flash externas são usadas como backup do sistema entre outras aplicações como
armazenamentos de dados científicos antes do seu envio para a terra. Espera-se que
esta operação seja rápida e não corrompida, já que o envio destes dados e crucial para
o objetivo final da missão.

\chapter{Resultados}\label{cap:resultados}
\section{Introdução}

Os dados de temperatura coletados foram salvos em um arquivo csv, testados com o algoritmo de intervalo de confiança móvel ajustado para avaliar o dado em uma amostra dos últimos 10 valores. 
Na tabela~\ref{tab:tamanhovetor} mostra a quantidade de dados que foram considerados validos em comparação a quantidade de últimas amostras coletadas, o algoritmo visa rodar em ambiente embarcado limitado onde quantidade de memoria e processamento são escassos, com isso o tamanho do vetor pode ser uma variável significativa na tomada de decisão. O vetor escolhido apresenta 10 casas de amostra e suas características foram descritas logo em seguida.

\begin{longtable}{|p{4cm}|p{3.5cm}|}
    \hiderowcolors
    \caption{Tamanho do vetor de amostra versus quantidade de dados filtrados}
    \label{tab:tamanhovetor}\\
    \showrowcolors
    \hline
    \rowcolor[HTML]{C0C0C0} 
    \multicolumn{1}{c|}{\cellcolor[HTML]{C0C0C0}\textbf{TAMANHO}} & \multicolumn{1}{c|}{\cellcolor[HTML]{C0C0C0}\textbf{DADOS FILTRADOS}} \\ \hline

    \endfirsthead
    \rowcolor[HTML]{C0C0C0} 
    \multicolumn{1}{c|}{\cellcolor[HTML]{C0C0C0}\textbf{TAMANHO}} & \multicolumn{1}{c|}{\cellcolor[HTML]{C0C0C0}\textbf{DADOS FILTRADOS}} \\ \hline

    \endhead
		\hline
		5	& 619	\\
		\hline
		10	& 404	\\
		\hline
		20	& 371	\\
		\hline
		30	& 383	\\
		\hline
		40	& 393	\\
		\hline
		50	& 385	\\
		\hline
		60	& 343	\\
		\hline
		70	& 296	\\
		\hline
		80	& 277	\\
		\hline
		90	& 266	\\
		\hline
		100	& 261	\\
		\hline
		200	& 391	\\
		\hline
    
    \end{longtable}

E importante ressaltar que em ambiente de sistema embarcado, a quantidade de memoria disponível para armazenar um vetor com muitas posições e pequena. Com isso, se torna um ponto positivo o sistema que consiga entregar uma filtragem aceitável com pouca alocação de memoria. 

% Os resultados podem ser visualizados na Figura~\ref{fig: Cleaning_sensor_data_overlay_visualization}, onde os pontos verdes significam que foram coletados e os em vermelhos foram ignorados. Já no gráfico ..
Na Figura~\ref{fig: graficos_tamanhos_amostras} podemos visualizar como o algoritmo se comporta de acordo com o tamanho do vetor de amostras, percebe-se que quanto maior o vetor menor a taxa de dados que passam pelo filtro, também presenciamos uma pior leitura nos picos de sinais já que o mesmo só começa a considerar os valores validos quando a curva já está em queda. O vetor com 5 amostra tem um período menor para considerar os novos valores, com isso muitos dos valores foram considerados limpos, isso pode ser bom visando que o tempo de coleta seja curto, mas em casos que deseja-se uma maior precisão um período maior de coleta parece ser mais adequado. 


\begin{figure}[H]
	\centering
	\includegraphics[width=15cm]{imagens/sensores/graficos_tamanhos_amostras.jpg}
	\caption{Gráficos mostrando a relação de tamanho da amostra versos quantidade de dados filtrados}
	Fonte: Autor
	\label{fig: graficos_tamanhos_amostras}
\end{figure}

Com 10 valores percebe-se que o algoritmo consegue pegar bem a ponta dos picos de temperatura, ignorando muitos dos ruídos mas com uma boa taxa de validação de dados, por isso levamos em consideração essa quantidade de amostras no teste, mas vale lembrar que essa quantidade pode e deve ser customizada de acordo com o problema, limitações e tempo em que a aplicação ira existir.

Nota-se a dificuldade do algoritmo em considerar os dados que estão em uma curva ascendente ou decrescente, mas destacam-se por conseguir capturar os dados da ponta de todos os grandes picos de sinal, em todos dos casos o tempo em que se levou para uma nova coleta depois de uma queda ou subida brusca e considerável, deixando claro que em caso de uma curva muito longa o programa pode travar esperando uma resposta, por isso e de importância avaliar se deve haver um controle de chamada do valor mesmo que esteja fora do intervalo mas com um tempo já decorrido consideravelmente.


\begin{figure}[H]
	\centering
	\includegraphics[width=15cm]{imagens/sensores/filtrado_100_ultimas.png}
	\caption{Dados de 1530 a 1590}
	Fonte: Autor
	\label{fig: indice}
\end{figure}

No trecho da figura~\ref{fig: indice} temos o trecho das primeiras 100 linhas, podemos notar como o algoritmo coleta os dados de forma precisa por mais que o desvio dos valores não seja grande, esses dados podem ser tratados posteriormente com uma simples média móvel ou outra função de preferencia.


\begin{figure}[H]
	\centering
	\includegraphics[width=15cm]{imagens/sensores/indice2.png}
	\caption{Curva de temperatura no trecho de 280 a 400}
	Fonte: Autor
	\label{fig: indice2}
\end{figure}

Na Figura~\ref{fig: indice2} o programa mostra sua falha em capturar curvas expressivas de dados, conseguindo capturar os valores do topo mas ignorando grande parte dos valores. Essa característica indesejável não parece ser corrigida com nenhum tamanho de amostra utilizado aqui neste trabalho, com todas falhando em lidar com sinuosidades.

\begin{figure}[H]
	\centering
	\includegraphics[width=15cm]{imagens/sensores/indice4.png}
	\caption{Conjunto de dados de 130 a 200}
	Fonte: Autor
	\label{fig: indice4}
\end{figure}

Na Figura~\ref{fig: indice4} podemos visualizar bem alguns ruídos que se destacam dos dados originais e como o algoritmo se comporta na sua filtragem, os valores que chegam a ter \ang{17}c de diferença são descartados já que apresentam características de ruído pela sua dessemelhança em tão pouco tempo.

\begin{figure}[H]
	\centering
	\includegraphics[width=15cm]{imagens/sensores/indice3.png}
	\caption{Dados de 800 a 950}
	Fonte: Autor
	\label{fig: indice3}
\end{figure}

Já na Figura~\ref{fig: indice3} presenciamos ruídos discrepantes de forma muito negativa e muito positiva, nota-se que os dados de 920 a 950 podem ser considerados ruídos pelo seu curto espaço de tempo, mas pela quantidade de amostra ser pequena alguns dos valores entraram como verdadeiros, enfatizando ainda mais que o tamanho do vetor de amostra deve ser dimensionado de acordo com os propósitos de precisão da aplicação.

Durante a execução do algoritmo notou-se o problema de laço teoricamente infinito caso não seja encontrado um valor dentro do intervalo de confiança, esse tempo pode ser prejudicial ou não dependendo da aplicação, então aconselha-se que seja definido um tempo limite para coleta do dado mesmo que esteja fora do intervalo de confiança. 

\begin{algorithm}[H]
    \Entrada{Tamanho do vetor de amostras ($TV$), Vetor de amostras ($VA$), Tempo limite de tentativas ($T$)}
    \Saida{Valor considerado verdadeiro ($resultado$), Vetor de amostras ($VA$)}
    \Inicio{
		\While{$resultado == SemValor$}{

			$V \leftarrow coletaDadoDoSensor()$; \tcc*[f]{Coleta dado do sensor} \\

			\Se{$T \leq 0$}{ 
				$resultado \leftarrow V$;
			}
			$T \rightarrow 1$;  \tcc*[f]{Decrementa tempo} \\

			\Se{$tamanho(VA) \geq TV$}{ 

				$primeiroIntervalo$, $segundoIntervalo \leftarrow intervaloDeConfianca($VA$)$; 

				\Se{$V \geq primeiroIntervalo$ \&\& $V \leq segundoIntervalo$}{
					$resultado \leftarrow V$; 
				}
				
				$VA \rightarrow primeiroElemento$; \tcc*[f]{Remove elemento mais velho} \\
			}

			$VA \leftarrow V$; \tcc*[f]{Adiciona valor para dentro do vetor de amostra} \\
		}


		\tcc*[f]{Retorna valor verdadeiro e vetor com as ultimas amostras} \\
		\Retorna $resultado$, $VA$; 
    }
    \caption{Algoritmo que considera o tempo na coleta do sensor}
    \label{algoritmo:alg_com_temp}
\end{algorithm}

Principalmente em ambiente embarcado, onde as rotinas de interrupção podem interromper a leitura do sensor em um tempo critico, ou as rotinas de seguranças podem reiniciar o sistema por culpa da característica de loop infinito ou de bloqueio do sistema advindo da função utilizada.

\chapter{Conclusão}\label{cap:conclusao}

Os sensores fazem parte da revolução tecnológica que o mundo tem passado, os dados coletados pelos mesmos são utilizados nas mais diversas esferas, são importantíssimos para as tomadas de decisão e estão sempre sujeitos a interferências.
 
Espera-se que todo sinal vindo de qualquer sensor esteja com algum ruído, e que sejam tratados de alguma forma. Para isso existem diversas funções para se retirar valores discrepantes, muitas delas interferem na forma do sinal real tratando o dado e retirando um novo valor resultante e de maior precisão, essa ação pode ser prejudicial em muitos do casos alterando o real valor e dificultando ou atrasando a visualização de mudanças bruscas, muito importantes em algumas analise de dados. No ambiente de sistemas embarcados é muito importante que qualquer algoritmo respeite as limitações de memória, processamento e tempo. Algumas funções podem exigir muito processamento ou até mesmo muita memória para realizar todos os cálculos responsáveis pela filtragem, com esse problema é importante que a função esteja dentro dos parâmetros para trabalhar em ambiente restrito.
Este trabalho teve como objetivo apresentar o desvio de confiança como uma boa alternativa para ser utilizada como filtro em ambiente embarcado, além disso era importante que os dados filtrados fossem os mais originais possíveis. O resultado foi um algoritmo que trabalhou bem com um pequeno vetor de amostra móvel, que manteve os dados de forma original mas com o crivo dos valores dentro de um intervalo aceitável de confiança. Mas observou-se problemas importantes, como não conseguir registrar sinais em curvas muito prolongadas, o que tornava a espera por um novo sinal confiável muito longa e atrapalhava o andamento do programa, por mais que o algoritmo tenha se saído bem em retornar os valores superiores de todos os picos de valores. Também notou-se que os dados resultantes por serem intocáveis, entregavam a possibilidade do método ser utilizado em conjunto com outros que posteriormente coletariam os dados, filtrando-os em busca de valores mais amenizados dependendo da aplicação.
 
O tempo envolvido neste trabalho não foi o suficiente para a realização de todos os cálculos em linguagem C em um sistema embarcado, mas o tratamento provou-se ser viável e poderia ser aplicado em conjunto com outros métodos em trabalhos futuros. Retornando à comunidade de desenvolvimento de filtros digitais em sistemas embarcados uma alternativa aberta a ser utilizada.




% O mercado de pequenos satélites e de extrema importância para qualquer pais, 
% universidade ou empresa que deseja entrar no ramo espacial, já que o mesmo 
% representa uma grande parcela dos lançamentos com um custo relativamente baixo 
% mas com um grande potencial de avanço tecnológico.

% Esperasse que o Zephyr seja capaz de suportar uma missão de CubeSat completa 
% já que o mesmo é bastante completo. Os testes aqui presentes buscam validar se seu 
% tempo de resposta e compatível com outro RTOS já usado no mercado, com essa 
% contribuição aguardamos que as equipes possam ter outra oferta de peso para 
% desenvolvimento, com uma comunidade grande e com possibilidade de ser ainda 
% maior, como também provando a qualidade do processador ESP32 em ser versátil 
% a qualquer sistema operacional do mercado, com sua rápida adoção pela comunidade. 
% Os testes deste trabalho não devem ser usados exclusivamente para 
% a escolha correta de um RTOS, já que se deve levar em consideração muitos 
% outros fatores não abordados neste trabalho, já que projetos de espaçonaves 
% aderem muitos outros campos, não sendo exclusividade somente do tempo como 
% uma variável exclusivamente importante. 
% Também com o estudo gerado, esperasse que o mesmo seja usado em outras 
% plataformas fora o ESP32, atribuindo métricas importantes para a escolha de um 
% RTOS novo em uma missão. 

% Missões espaciais tem muitos critérios, este trabalho deseja ser o ponta pé em 
% outros trabalhos que também possam avaliar e definir novas métricas de escolha, 
% que facilitem e deem tempo a equipes de construção, já que o principal objetivo 
% de um pequeno satélite e ser barato, seu tempo de desenvolvimento conta muito em 
% seu custo final.


\section{Trabalhos Futuros}

\begin{itemize}
    \item Utilizar o filtro apresentado com outras funções de filtragem.
    \item Escrever o filtro em uma aplicação real de coleta de dados.
  \end{itemize}



% ----------------------------------------------------------
% ELEMENTOS PÓS-TEXTUAIS
% ----------------------------------------------------------
\postextual
% ----------------------------------------------------------

% ----------------------------------------------------------
% Referências bibliográficas
% ----------------------------------------------------------
%\bibliography{bibliografia}
\bibliography{bibliografia}
% ----------------------------------------------------------
% Glossário
% ----------------------------------------------------------
%
% Consulte o manual da classe abntex2 para orientações sobre o glossário.
%
%\glossary

%---------------------------------------------------------------------
% INDICE REMISSIVO
%---------------------------------------------------------------------
\phantompart
\printindex
%---------------------------------------------------------------------

\end{document}
\grid
