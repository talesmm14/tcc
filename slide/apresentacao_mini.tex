%% abtex2-modelo-slides.tex, v-1.0 gfabinhomat
%% Copyright 2012-2016 by abnTeX2 group at http://www.abntex.net.br/ 
%%
%% This work may be distributed and/or modified under the
%% conditions of the LaTeX Project Public License, either version 1.3
%% of this license or (at your option) any later version.
%% The latest version of this license is in
%%   http://www.latex-project.org/lppl.txt
%% and version 1.3 or later is part of all distributions of LaTeX
%% version 2005/12/01 or later.
%%
%% This work has the LPPL maintenance status `maintained'.
%% 
%% The Current Maintainer of this work is Fábio Rodrigues Silva, 
%% member of abnTeX2 team, led by Lauro César Araujo. 
%% Further information are available on 
%% http://www.abntex.net.br/
%%
%% This work consists of the files abntex2-modelo-slides.tex, 
%% abntex2-modelo-references.bib and abntex2-modelo-marca.pdf
%%
%% Modelo desenvolvido por Fábio Rodrigues Silva (gfabinhomat@gmail.com)
%% Mais informações podem ser obtidas no guia do usuário Beamer 
%% (http://linorg.usp.br/CTAN/macros/latex/contrib/beamer/doc/beameruserguide.pdf)
%% Informações rápidas podem ser acessadas em http://en.wikibooks.org/wiki/LaTeX/Presentations


% Apresentações em widescreen. Outros valores possíveis: 1610, 149, 54, 43 e 32.
% Por padrão, as apresentações são no formato 4:3 (sem o aspectratio).
%\documentclass[aspectratio=169]{beamer}	 	
\documentclass[]{beamer}	 	

\usetheme{Berlin}
\usecolortheme{default}
\usefonttheme[onlymath]{serif}			% para fontes matemáticas
% Enconte mais temas e cores em http://www.hartwork.org/beamer-theme-matrix/ 
% Veja também http://deic.uab.es/~iblanes/beamer_gallery/index.html

% Customizações de Cores: fg significa cor do texto e bg é cor do fundo
\setbeamercolor{normal text}{fg=black}
\setbeamercolor{alerted text}{fg=red}
\setbeamercolor{author}{fg=blue}
\setbeamercolor{institute}{fg=blue}
\setbeamercolor{date}{fg=green}
\setbeamercolor{frametitle}{fg=white}
\setbeamercolor{framesubtitle}{fg=white}
\setbeamercolor{block title}{bg=blue, fg=white}		%Cor do título
\setbeamercolor{block body}{bg=gray, fg=darkgray}	%Cor do texto (bg= fundo; fg=texto)
\setbeamerfont{title}{family=\rmfamily,series=\bfseries,size={\fontsize{14}{16}}}
%\setbeamertemplate{caption}[numbered]
%\setbeamertemplate{bibliography item}{[\theenumiv]}
\setbeamertemplate{bibliography item}{}

% ---
% PACOTES
% ---
\usepackage[alf]{abntex2cite}		% Citações padrão ABNT
\usepackage[brazil]{babel}		% Idioma do documento
\usepackage{color}			% Controle das cores
\usepackage[T1]{fontenc}		% Selecao de codigos de fonte.
\usepackage{graphicx}			% Inclusão de gráficos
\usepackage[utf8]{inputenc}		% Codificacao do documento (conversão automática dos acentos)
\usepackage{txfonts}			% Fontes virtuais
\usepackage{subfig}
\usepackage{sidecap}
\usepackage{verbatim}
\graphicspath{{imagens/}}




\usepackage{lmodern}			% Usa a fonte Latin Modern			
\usepackage[utf8]{inputenc}		% Codificacao do documento (conversão automática dos acentos)
\usepackage{lastpage}			% Usado pela Ficha catalográfica
\usepackage{indentfirst}		% Indenta o primeiro parágrafo de cada seção.



\usepackage{subfig}
\usepackage{graphicx}			% Inclusão de gráficos

\usepackage{microtype} 			% para melhorias de justificação
\usepackage{soul}
\usepackage{amssymb}
\usepackage{amsmath}
\usepackage{spreadtab}
\usepackage{multirow}
\usepackage{amsthm}
\usepackage{url}


\usepackage[portuguese, ruled, linesnumbered]{algorithm2e}

\usepackage{algorithmic}

% Utilizado para customizar as fontes das equações 
\usepackage{fixmath}
\usepackage{verbatim}

%\usepackage[brazilian,hyperpageref]{backref}	 % Paginas com as citações na bibl


\graphicspath{{imagens/}}


\usepackage{longtable}
%\usepackage{csvsimple,booktabs}
\usepackage{pgfplotstable}
\usepackage{filecontents}
%\usepackage{tabu}
\usepackage{makecell}
\usepackage{booktabs}
\usepackage{float}

\usepackage{multirow}
%\usepackage[table]{xcolor}

%\usepackage{colortbl}
%\definecolor{lightgray}{gray}{0.9}

\usepackage[brazil]{babel} % for European Portuguese use portuguese
\usepackage[fixlanguage]{babelbib}
\selectbiblanguage{brazil}
\citeoption{abnt-etal-cite=2}
% ---


\title{Modelagem mono e multiobjetivo para o 
problema de sequenciamento de tarefas em 
máquinas paralelas não relacionadas}
\author[Hugo Cavalcante Lima]{Hugo Cavalcante Lima \\ Orientador: Prof. Me. Douglas Chagas da Silva
{\\ \footnotesize\ttfamily yugolemom@gmail.com}}
\institute{%
	Universidade Estadual do Tocantins - UNITINS
	%\par 
	\\
	Curso de Sistemas de Informação
}
\date{2018}



% --- Informações do documento ---

% ---

% ----------------- INÍCIO DO DOCUMENTO --------------------------------------
\begin{document}

% ----------------- NOVO SLIDE --------------------------------
\begin{frame}

\begin{minipage}{1\linewidth}
  \centering
  \center
  \includegraphics[width=0.4\textwidth]{imagens/unitins.png}
 
\end{minipage}

\titlepage

\end{frame}

% ----------------- NOVO SLIDE --------------------------------
\begin{frame}{Sumário}
\tableofcontents
\end{frame}

% ----------------- NOVO SLIDE --------------------------------
\section{Introdução}
\begin{frame}{Introdução}
%\framesubtitle{Otimização}
	\begin{itemize}
		\item Melhorar o processo produtivo, de modo a conseguir eficiência e produtividade;
		\item Problemas mono-objetivo e multiobjetivo;
		\item Otimização de equipamentos e ativos, visando lucratividade;
		\item Organizar o sequenciamento e otimizar objetivos conflitantes.
	\end{itemize}
\end{frame}

\begin{comment}
	\begin{frame}{Motivação}

		Desenvolver uma modelagem para o problema e a aplicação dos conceitos de otimização,
		possibilitando uma melhor utilização dos recursos no processo de fabricação.
			
	\end{frame}
\end{comment}
% ----------------- NOVO SLIDE --------------------------------

\begin{frame}{Objetivo Geral}
	
	Elaborar uma formulação matemática para os problemas e desenvolver algoritmos mono-objetivos 
	e multiobjetivos para o problema de sequenciamento de máquinas paralelas não relacionadas.

\end{frame}

\begin{frame}{Objetivos Específicos}

	\begin{itemize}
		\item Compreender os conceitos envolvendo problemas de sequenciamento de máquinas;
		\item Propor a formulação matemática para o problema abordado;
		\item Implementar algoritmos mono-objetivos e multiobjetivos para a resolução do problema;
		\item Demonstrar os resultados objetivos.
	\end{itemize}
	
\end{frame}

% ----------------- NOVO SLIDE --------------------------------

\section{Referencial Teórico}
\begin{frame}{Conceitos de Otimização}

	\begin{itemize}
	\item Buscar as melhores soluções para o aproveitamento de recursos;
	\item Baseado nas variáveis do Projeto. 
	\end{itemize}

	\begin{equation}\label{equacao:otimizacao}
		{ arg \ {min/max} \ f_{obj}(x)  }
	\end{equation}
	de modo que:
	\begin{equation}\label{equacao:otimizacao_II}
		{ g_{i}(x) \leq 0 \ para \ i = 1,...,n_{g} }
	\end{equation}
	\begin{equation}\label{equacao:otimizacao_III}
		{ h_{j}(x) \leq 0 \ para \ i = 1,...,n_{h} }
	\end{equation}
	\begin{equation}\label{equacao:otimizacao_IV}
		{ lb_{k} \leq x_{k} \leq ub_{k} \ para \ i = 1,...,n_{x} }
	\end{equation}

\end{frame}
\begin{comment}
	\begin{frame}{Conceitos de Otimização}
		\framesubtitle{Região de busca e Ponto ótimo}

		\begin{figure}[!htb]
			\begin{center}
				\includegraphics[scale=0.5]{imagens/grafico_ponto_busca}
			\end{center}
			\label{figura:grafico_ponto_busca}
			\caption{Região de busca e Ponto ótimo - (AZUMA, 2011)}	
		\end{figure}
		
	\end{frame}	
\end{comment}
\begin{frame}{Conceitos de Otimização}
	\framesubtitle{Metas da otimização multiobjetivo}
	
	\begin{enumerate}
		\item Alcançar um conjunto de soluções que esteja o mais próximo da fronteira de Pareto;
		\item Atingir um conjunto de soluções com a maior diversidade de valores possíveis;
		\item Conseguir atingir as outras duas metas anteriores com o menor gasto de processamento computacional possível.
	\end{enumerate}
		
\end{frame}

\begin{frame}{Conceitos de Otimização}
	\framesubtitle{Metas da otimização multiobjetivo}
	
	\begin{figure}[!htb]	
		\begin{center}
			\includegraphics[scale=0.6]{fronteira_pareto}
		\end{center}
		\caption{Metas da otimização multiobjetivo - (AZUMA, 2011)}
		\label{figura:fronteira_pareto}
	\end{figure}	

\end{frame}

% ----------------- NOVO SLIDE ---------------------- %


\begin{frame}{Métodos de Otimização}
	
	Os Métodos são classificados conforme a função objetivo e suas restrições e fundamental
	escolher a técnica mais apropriada ao problema.	
	
	\begin{itemize}
		\item Programação linear
		\item Programação não linear
		\item Programação linear inteira
	\end{itemize}	
\end{frame}
\begin{comment}
	\begin{frame}{Métodos de Otimização}
		
		\begin{table}[!htb]
			\centering
			\label{tab:programacao_linear_x_nao_linear}
			\resizebox{\textwidth}{!}{%
			\begin{tabular}{|l|l|l|}
			\hline
			\multicolumn{1}{|c|}{\textbf{Quesito}} & \multicolumn{1}{c|}{\textbf{Programação linear}} & \multicolumn{1}{c|}{\textbf{Programação não-linear}} \\ \hline
			Representação do problema & \begin{tabular}[c]{@{}l@{}}Restrita, porque não considera \\ aspectos causadores de não \\ linearidade, tais como:eficiência \\ e ineficiência de produtos em \\ escalas diferentes, efeitos da \\ quantidade de venda nos preços \\ unitários.\end{tabular} & \begin{tabular}[c]{@{}l@{}}Abrangente à medida que tenta\\ incorporar esses aspectos\\ desconsiderandos no modelo linear.\end{tabular} \\ \hline
			Nível de complexidade & \begin{tabular}[c]{@{}l@{}}Simplificado, devido à abordagem \\ restrita do problema.\end{tabular} & \begin{tabular}[c]{@{}l@{}}Complexo,em virtude da riqueza e \\ abrangência abordada.\end{tabular} \\ \hline
			Custo de processamento & Baixo & Alto \\ \hline
			Aplicabilidade & \begin{tabular}[c]{@{}l@{}}Quando o problema tem limitada \\ área de soluções possíveis e existe \\ boa noção sobre o posicionamento \\ da solução ótima, possibilitando \\ aproximação linear adequada.\end{tabular} & \begin{tabular}[c]{@{}l@{}}Ao contrário, quando o problema\\ tem ampla área de soluções \\ possíveis e inexiste boa noção sobre \\ o possicionamento da solução ótima,\\ dificultando aproximação linear\\ adequada.\end{tabular} \\ \hline
			\begin{tabular}[c]{@{}l@{}}Nível de cautela na análise\\ dos resultados\end{tabular} & Alto & \begin{tabular}[c]{@{}l@{}}Baixo, devido ao maior esforço de incluir\\ os aspectos que causam não-linearidade.\end{tabular} \\ \hline
			\end{tabular}%
			}
			\caption{Programação linear x não-linear - (CORRAR, 2004)}
		\end{table}    
	\end{frame}


\begin{frame}{Métodos de Otimização}
	\framesubtitle{Programação Linear Inteira e Binária}
	\begin{itemize}
		\item Quando todas as variáveis do problema pertencem ao conjunto dos números inteiros;
		\item Problemas onde existe uma série de decisões ("sim" ou "não").
	\end{itemize}
	\centering
	\textbf{"É viável um investimento fixo ?"}

	\begin{equation}\label{equacao:programacao_inteira}
		x_j = \begin{cases}0 & se\ a \ decisao\ \mathbold{j} \ for \ nao\\1 & se\ a \ decisao\ \mathbold{j} 
			\ for \ sim\ 
		\end{cases}
	\end{equation}

\end{frame}
\end{comment}
%\begin{frame}{Métodos de Otimização}
%	\framesubtitle{Programação Linear Inteira e Binária}
	
%		Problemas considerados difíceis, deste modo é recomendado, recorrer 
%		a soluções que possam percorrer os espaços de soluções de forma cautelosa.
%\end{frame}

\begin{comment}
	\begin{frame}{Métodos de Otimização}
		\framesubtitle{Método Branch and Bound}

		\begin{itemize}
			\item Método emprega o conceito básico de dividir e conquistar;
			\item \textbf{Particionamento,Poda}.
		\end{itemize}

	\end{frame}
\end{comment}

\begin{comment}
	\begin{frame}{Métodos de Otimização}
		\framesubtitle{Método Branch and Bound}

		\begin{figure}
			\begin{minipage}{.5\textwidth}
					\centering
					\includegraphics[scale=0.32,keepaspectratio=true]{limites_branch_bound}
					%\caption{Limites no método e a solução ótima}
					\label{figura:limites_branch_bound}
			\end{minipage}%
			\begin{minipage}{.5\textwidth}
					\centering
					\includegraphics[scale=0.35,keepaspectratio=true]{representacao_branch_bound}
					%\caption{Particionamento no método}
					\label{figura:representacao_branch_bound}
			\end{minipage}
			\caption{Branch and Bound - (ALBUQUERQUE, 2011)}	
		\end{figure}

	\end{frame}

	\begin{frame}{Métodos de Otimização}
		\framesubtitle{Método Branch and Bound}
		\begin{itemize}
			\item $z_{i} = U$ Solução ideal foi processada acontece a poda por resolução;
			\item $z_{i} > U$ O limite inferior e maior do que o limite global, poda por limitação;
			\item $z_{i} < U$ O ramo pode conter uma solução, acontece o particionamento;
			\item Diante de um subproblema com solução impraticável ocorre a poda.
		\end{itemize}
	\end{frame}
\end{comment}

\begin{comment}
	\begin{frame}{Métodos de Otimização}
		\framesubtitle{Programação de tarefas em máquinas paralelas}
		\begin{itemize}
			\item Programação de tarefas (\textit{job scheduling});
			\item Máquinas paralelas (idênticas,uniformes e não relacionadas).
		\end{itemize}	
	\end{frame}

	\begin{frame}{Métodos de Otimização}
		\framesubtitle{Programação de tarefas em máquinas paralelas}
		\begin{itemize}
			\item Atraso total das tarefas (\textit{makespan});
			\item Atraso total ponderado (\textit{tardiness});
			\item Adiantamento total ponderado (\textit{lateness});
			\item Soma ponderada de atrasos e adiantamentos.
		\end{itemize}	
	\end{frame}

\end{comment}

\section{Metodologia}

	\begin{frame}{Metodologia}
		\framesubtitle{Problema de máquina paralelas não relacionadas}
		\begin{itemize}
			\item As entradas do problema foram construídas de maneira determinística conforme sugerido pelos autores;
		\end{itemize}	
			Variáveis iniciais do problema:
		\begin{itemize}
			\item A data de entrega de cada tarefa ($t_{ij} $ ): $6$;
    		\item O número de máquinas ($M$): $5$;
   			\item O número de tarefas ($N$): $25$. 
		\end{itemize}
	\end{frame}

	\begin{frame}{Formulação Matemática Proposta}	
		Problema de minimização do tempo total de entrega (\textit{makespan}):
		\begin{equation}\label{equacao:minimizacao_tempo_entrega}
			\begin{cases}
				1. \ \ c_{1} = min \ Z  \\  
				2. \ \ \sum_{i=1}^M  \sum_{j=1}^N  \ t_{ij} \cdot x_{ij} \leq c_{1} \\
				3. \ \ \sum_{i=1}^M  \ x_{ij} = 1,j = 1, ...,N  \\
				4. \  \ x_{ij} = 0 \ ou \ 1,i = 1, ...,m;\ j = 1, ...,N 
			\end{cases}
		\end{equation}
	\end{frame}

	\begin{frame}{Formulação Matemática Proposta}	
		O atraso $L_{j}$ de cada tarefa e apresentado pela Equação \ref{equacao:atraso_tarefa}:

		\begin{equation} \label{equacao:atraso_tarefa}
			L_{j} = max(C_{j} - d_{j},0)
		\end{equation}
	\end{frame}

	\begin{frame}{Formulação Matemática Proposta}	
		O adiantamento $T_{j}$ de cada tarefa e apresentado pela Equação \ref{equacao:adiantamento_tarefa}:

		\begin{equation}\label{equacao:adiantamento_tarefa}
			T_{j} = max(d_{j} - C_{j},0)
		\end{equation}
	\end{frame}

	\begin{frame}{Formulação Matemática Proposta}	
		O Conjunto de Equações \ref{equacao:minimizacao_soma_atrasos} para
    resolução do problema de minimização da soma ponderada dos atrasos: 


		\begin{equation}\label{equacao:minimizacao_soma_atrasos}
			\begin{cases}
				1.\ \   c_{2} = min \ (\sum_{j=1}^N \ w_{j}   \cdot L_{j} ) \\ 
				2.\ \   \sum_{i=1}^M  \sum_{j=1}^N  \ (w_{j} \cdot L_{j}) \cdot x_{ij} \leq c_{2} \\
				3.\ \   \sum_{i=1}^M  \ x_{ij} = 1,j = 1, ...,N \\
				4.\ \   x_{ij} = 0 \ ou \ 1,i = 1, ...,m;\ j = 1, ...,N
			\end{cases}
		\end{equation}
	\end{frame}

	\begin{frame}{Formulação Matemática Proposta}	
		O Conjunto de Equações \ref{equacao:minimizacao_soma_adiantamentos} para
    resolução do problema de minimização da soma ponderada dos adiantamentos: 


		\begin{equation}\label{equacao:minimizacao_soma_adiantamentos}
			\begin{cases}
			1. \ \ c_{3} = min \ (\sum_{j=1}^N \ w_{j}   \cdot T_{j} ) \\ 
			2. \ \ \sum_{i=1}^M  \sum_{j=1}^N  \ (w_{j} \cdot T_{j})  \cdot x_{ij} \leq c_{3} \\
			3. \ \ \sum_{i=1}^M  \ x_{ij} = 1,j = 1, ...,N \\
			4. \ \ x_{ij} = 0 \ ou \ 1,i = 1, ...,m;\ j = 1, ...,N
			\end{cases}
		\end{equation}

	\end{frame}

	\begin{frame}{Formulação Matemática Proposta}	
		O Conjunto de Equações \ref{equacao:minimizacao_soma_atrasos_adiantamentos} para
		resolução do problema de minimização da soma ponderada de atrasos e adiantamentos: 
	
	
		\begin{equation}\label{equacao:minimizacao_soma_atrasos_adiantamentos}
			\begin{cases}
				1.\ \  c_{4} = min \ (\sum_{j=1}^N \ w_{j}   \cdot (L_{j} \cdot T_{j}) ) \\ 
				2.\ \  \sum_{i=1}^M  \sum_{j=1}^N  \ (w_{j} \cdot (L_{j} + T_{j}))  \cdot x_{ij} \leq c_{4} \\
				3.\ \  \sum_{i=1}^M  \ x_{ij} = 1,j = 1, ...,N \\
				4.\ \  x_{ij} = 0 \ ou \ 1,i = 1, ...,m;\ j = 1, ...,N
			\end{cases}
		\end{equation}
	\end{frame}

	\begin{frame}{Formulação Matemática Proposta}	
		Agrupamento de equações para os problemas multiobjetivos:

		O Conjunto de Equações \ref{equacao:makespan_minimizacao_soma_atrasos_adiantamentos} para
		resolução do problema da minimização do \textit{makespan} $\times$ minimização da soma
		ponderada dos atrasos e adiantamentos: 


		\begin{equation}\label{equacao:makespan_minimizacao_soma_atrasos_adiantamentos}
			\begin{cases}
			1.\ \  c_{1} = min \ Z \\
			2.\ \  c_{4} = min  \ \left (  \sum_{j=1}^N  \ w_{j} \cdot (L_{j} + T_{j})\right ) \\
			3.\ \  \sum_{i=1}^M  \  \sum_{j=1}^N  \ (w_{i} \cdot T_{j} )\ x_{ij} \leq c_{1} \\
			4.\ \  \sum_{i=1}^M  \  \sum_{j=1}^N  \ (w_{i} \cdot T_{j} )\ x_{ij} \leq c_{4} \\
			5.\ \  \sum_{i=1}^M  \ x_{ij} = 1,j = 1, ...,N \\
			6.\ \  x_{ij} = 0 \ ou \ 1,i = 1, ...,m;\ j = 1, ...,N
			\end{cases}
		\end{equation}

	\end{frame}

	\begin{frame}{Formulação Matemática Proposta}	
			O Conjunto de Equações \ref{equacao:minimizacao_soma_atrasos_minimizacao_soma_adiantamentos} para
		resolução do problema da minimização da soma ponderada dos atrasos $\times$
		minimização da soma ponderada dos adiantamentos: 

		\begin{equation}\label{equacao:minimizacao_soma_atrasos_minimizacao_soma_adiantamentos}
			\begin{cases}
			1. \ \    c_{2} = min  \ \left (  \sum_{j=1}^N  \ w_{j} \cdot L_{j}\right ) \\
			2. \ \    c_{3} = min  \ \left (  \sum_{j=1}^N  \ w_{j} \cdot T_{j}\right ) \\
			3. \ \     \sum_{i=1}^M  \  \sum_{j=1}^N  \ (w_{i} \cdot T_{j} )\ x_{ij} \leq c_{2} \\
			4. \ \     \sum_{i=1}^M  \  \sum_{j=1}^N  \ (w_{i} \cdot T_{j} )\ x_{ij} \leq c_{3} \\
			5. \ \     \sum_{i=1}^M  \ x_{ij} = 1,j = 1, ...,N \\
			6. \ \     x_{ij} = 0 \ ou \ 1,i = 1, ...,m;\ j = 1, ...,N
			\end{cases}
		\end{equation}

	\end{frame}

	\begin{frame}{Metodologia}
		\framesubtitle{Materiais}
		\begin{itemize}
			\item MATLAB\textsuperscript{\textregistered} R2017B;
			\item RStudio V1.0.53;
			\item Notebook HP\textsuperscript{\textregistered}  Pavilion dv4-2173nr,com processador 
			Intel\textsuperscript{\textregistered}  Core i5-430M de 2.26GHz,  8GB de memória RAM e sistema 
			Linux Ubuntu 16.04 64bits. 
		\end{itemize}
	\end{frame}


\begin{frame}{Algoritmos}
	\begin{itemize}
		\item Equações matemáticas foram implementadas utilizando o Matlab\textsuperscript{\textregistered};
		\item As saídas dos dados foram processadas pelo Software RStudio\textsuperscript{\textregistered}.
		\item Foram realizados cálculos estatísticos nos resultados para avaliação do desenvolvimento dos 
		métodos.
	\end{itemize}
\end{frame}

\begin{frame}[allowframebreaks]
	\scalebox{.9}{                        %new code
		\begin{algorithm}[H]              %new code
			\Entrada{Processamento Tarefa/Máquina em dias ($F$),Número de 
			Testes ($It$) e Interações ($Zt$), Máquinas ($M$), Tarefas ($N$), Coeficientes de desiqualdades($Aeq$ e $beq$), Limites ($lb$ e $ub$), var inteiras ($intcon$) }
			\Saida{Resultado vetor de soluções ótimas ($x$) e a Solução otimizada ($result$)}
			\Inicio{				
				$beq \leftarrow ones(N,1)$; // Inicializa o  vetor de desiqualdades
				
				$lb \leftarrow zeros(intcon,1)$; // Inicializa o vetor de Limites inferiores

				$ub \leftarrow ones(intcon,1)$; / Inicializa o vetor de Limites superiores

				$Aeq=kron(eye(N),ones(1,M))$; // Matriz com Coeficientes

				\Para{i de 1 até It }{
					\Para{j  de 1 até Zt }{
						$[x,result] = intlinprog(F,intcon,[ \ ],[ \ ],Aeq,beq,lb,ub)$; 
					}
				}
			}
			\label{algoritmo:makespan}
			\caption{Algoritmo para o atraso total das tarefas}
		\end{algorithm}
	}

	\scalebox{.8}{                        %new code
		\begin{algorithm}[H]              %new code
			\Entrada{($fl$) variável auxiliar e a variável com data de entrega de cada tarefa $dd$}
			\Saida{vetor de soluções ótimas ($x$) e a Solução otimizada ($result$)}
			\Inicio{
				// Criando novo vetor de tarefas com restrição de atrasos

				$fl=f-dd$; 
				
				\Para{j de 1 até intcon }{  // Laço que vai empregar a restricao do atraso de cada tarefa 
					\Para{i  de 1 até M }{
						$fl(M * (j-1) + i)= w(j) * fl( M * (j-1) + i)$;            
					}
				}
			
				$fl(fl<0) = 0$; // Remove todos os valores inferiores a zero. 
				  
				\Para{i de 1 até It }{
					\Para{j  de 1 até Zt }{
						$[x,result] = intlinprog(fl,intcon,[ \ ],[ \ ],Aeq,beq,lb,ub)$; 
					}
				}
			}
			\caption{Algoritmo para o atraso total ponderado}
			\label{algoritmo:tardiness}
		\end{algorithm}
	}

	\scalebox{.8}{                        %new code
		\begin{algorithm}[H]              %new code
			\Entrada{($fl$) variável auxiliar e a variável com data de entrega de cada tarefa $dd$}
			\Saida{vetor de soluções ótimas ($x$) e a Solução otimizada ($result$)}
			\Inicio{
						 
				// Criando novo vetor de tarefas com restrição de adiantamentos

				$ft=dd-f$; 
		
				// Laço que vai empregar a restricao do adiantamento de cada tarefa 
				
				\Para{j de 1 até intcon }{  
					\Para{i  de 1 até M }{
						$ft(M * (j-1) + i)= w(j) * ft( M * (j-1) + i)$;            
					}
				}
				  
				\Para{i de 1 até It }{
					\Para{j  de 1 até Zt }{
						$[x,result] = intlinprog(ft,intcon,[ \ ],[ \ ],Aeq,beq,lb,ub)$; 
					}
				}
			}
			\caption{Algoritmo para o adiantamento total ponderado }
			\label{algoritmo:lateness}
		\end{algorithm}
	}

	\scalebox{.8}{                        %new code
		\begin{algorithm}[H]              %new code
			\Entrada{($ftl$) variável auxiliar e a variável com data de entrega de cada
			tarefa $dd$}
			\Saida{Vetor de soluções ótimas ($x$) e a Solução otimizada ($result$)}
			\Inicio{
				$ftl=abs(f-dd)$; // Criando novo vetor com valores absolutos das tarefas
		
				// Laço que vai empregar a restrição a cada tarefa 
				
				\Para{j de 1 até intcon }{  
					\Para{i  de 1 até M }{
						$ftl(M * (j-1) + i)= w(j) * ftl( M * (j-1) + i)$;            
					}
				}
			
				\Para{i de 1 até It }{
					\Para{j  de 1 até Zt }{
						$[x,result] = intlinprog(ft,intcon,[ \ ],[ \ ],Aeq,beq,lb,ub)$; 
					}
				}
			}
			\caption{Algoritmo soma ponderada de atrasos e adiantamentos}
			\label{algoritmo:soma_atrasos_adiantamentos}
		\end{algorithm}
	}

	\scalebox{.8}{                        %new code
		\begin{algorithm}[H]              %new code
			\Entrada{($ftl$) variável auxiliar e a variável com data de entrega de cada
			tarefa $dd$}
			\Saida{Vetor de soluções ótimas ($x$) e a Solução otimizada ($result$)}
			\Inicio{			
				//funcao makespan
	
				$fm = fmakespan(f,N,M)$;
		
				//funcao soma adiantamentos e atrasos
	
				$ftl = fsomaAd(f,N,M,dd)$;
				
				\Para{i de 1 até It }{
					\Para{j  de 1 até Zt }{
						
						// Empregas as retrições da relação entre makespan, atrasos e adiantamentos 
	
						$fx=Restric(fm,ftl)$;
	
						$[x,result] = intlinprog(fx,intcon,[ \ ],[ \ ],Aeq,beq,lb,ub)$; 
					}
				}
			}
			\caption{Algoritmo para a soma ponderada de atrasos e adiantamentos x Makespan}
			\label{algoritmo:makespan_minimizacao_soma_atrasos_adiantamentos}
		\end{algorithm}
	}

	\scalebox{.8}{  
		\begin{algorithm}[H]                         %new code
			\Entrada{($ftl$) variável auxiliar e a variável com data de entrega de cada tarefa ($dd$)}
			\Saida{Vetor de soluções ótimas ($x$) e a Solução otimizada ($result$)}
			\Inicio{
				//funcões da soma ponderada dos atrasos e soma adiantamentos

				$ft = ftardiness(f,N,M,dd)$;

				$fl = flateness(f,N,M,dd)$;
				
				\Para{i de 1 até It }{
					\Para{j  de 1 até Zt }{
						
						// Empregas as retrições da relação entre atrasos e adiantamentos 

						$fx=Restric2(ft,fl)$;

						$[x,result] = intlinprog(fx,intcon,[ \ ],[ \ ],Aeq,beq,lb,ub)$; 
					}
				}
			}
			\caption{Algoritmo para a soma ponderada de atrasos x adiantamentos}
			\label{algoritmo:minimizacao_soma_atrasos_minimizacao_soma_adiantamentos}
		\end{algorithm}
	}
\end{frame}

\section{Resultados}
\begin{frame}[allowframebreaks]{Resultados}
	
	\begin{table}[!htb]
		\centering
		\label{my-label}
		\resizebox{\textwidth}{!}{%
		\begin{tabular}{|c|c|l|c|c|}
		\hline
		\multirow{2}{*}{} & \multicolumn{4}{c|}{Mono objetivo} \\ \cline{2-5} 
		 & \multicolumn{1}{l|}{Makespan} & Tardiness & \multicolumn{1}{l|}{Lateness} & \begin{tabular}[c]{@{}c@{}}Atrasos \\ e \\ adiantamentos\end{tabular} \\ \hline
		Otimização & \textbf{48} & \multicolumn{1}{c|}{\textbf{0}} & \textbf{0} & \textbf{112} \\ \hline
		\multirow{2}{*}{} & \multicolumn{4}{c|}{Multi objetivo} \\ \cline{2-5} 
		 & \multicolumn{2}{c|}{\begin{tabular}[c]{@{}c@{}}Makespan \\ x \\ Atrasos e adiantamentos\end{tabular}} & \multicolumn{2}{c|}{\begin{tabular}[c]{@{}c@{}}Tardiness \\ x \\ Lateness\end{tabular}} \\ \hline
		Pontos Pareto & \multicolumn{2}{c|}{\textbf{16}} & \multicolumn{2}{c|}{\textbf{8}} \\ \hline
		Ponto otímo & \multicolumn{2}{c|}{\textbf{{[}48,122{]}}} & \multicolumn{2}{c|}{\textbf{{[}0,0{]}}} \\ \hline
		Pontos \textgreater Freq. & \multicolumn{2}{c|}{\textbf{{[}150,112{]}}} & \multicolumn{2}{c|}{\textbf{{[}0,167{]}}} \\ \hline
		Qt x \textgreater Freq. & \multicolumn{2}{c|}{\textbf{13}} & \multicolumn{2}{c|}{\textbf{7}} \\ \hline
		\end{tabular}%
		}
	\end{table}

	\begin{table}[!htb]
		\centering
		\label{tab:multi_estatisticos}
		\begin{tabular}{|c|c|c|c|c|}
			\hline
			 & \multicolumn{2}{c|}{MULTI-1} & \multicolumn{2}{c|}{MULTI-2} \\ \hline
			Média & 117.1 & 186.9 & 88.2 & 55.04 \\ \hline
			Mediana & 125 & 128 & 80 & 37 \\ \hline
			Variância & 1126.139 & 16453.89 & 5333.306 & 3396.366 \\ \hline
			Desvio Padrão & 33.558 & 128.2727 & 73.02949 & 58.27835 \\ \hline
			\multirow{2}{*}{\begin{tabular}[c]{@{}c@{}}Intervalo de \\ Confiança\end{tabular}} & 107.5229 & 150.4453 & 67.44525 & 38.47748 \\ \cline{2-5} 
			 & \multicolumn{1}{l|}{126.5971} & \multicolumn{1}{l|}{223.3547} & \multicolumn{1}{l|}{108.95475} & \multicolumn{1}{l|}{71.60252} \\ \hline
			\textbf{Mínimo} & \textbf{48} & \textbf{112} & \textbf{0} & \textbf{0} \\ \hline
			Máximo & 150 & 521 & 167 & 143 \\ \hline
		\end{tabular}
		\caption{Dados estatísticos para os problemas multiobjetivos}
	\end{table}

	\begin{figure}
		\label{figura:tm_figura}
		\begin{center}
			\includegraphics[scale=0.32,keepaspectratio=true]{tm_figura}
		\end{center}
		\caption{Soma ponderada dos atrasos e adiantamentos x Makespan}
		%\legend{Fonte: Autor}
	\end{figure}

	\begin{figure}		
		\label{figura:tl_figura}	
		\begin{center}
			\includegraphics[scale=0.32,keepaspectratio=true]{tl_figura}
		\end{center}
		\caption{Soma ponderada dos adiantamentos x Soma ponderada dos atrasos}    
		%\legend{Fonte: Autor}
	\end{figure}

	
\end{frame}


\section{Conclusões}
\begin{frame}{Conclusões}
	\begin{itemize}
		\item Os algorítimos mostraram-se eficientes na distribuição das soluções na fronteira Pareto;
        \item A modelagem é genérica para qualquer tipo de entrada, considerando 5 máquinas, mais pode
        ser aplicada para qualquer quantidade de máquinas e tarefas;
		\item Os algorítimos implementados apresentaram baixo consumo computacional;
        \item A ToolBox nativa do Matlab proporcionou maior agilidade no desenvolvimento das soluções;
        \item As soluções demonstraram ser interessantes mais precisam ser melhoradas.
        	
	\end{itemize}
\end{frame}

\begin{frame}{Trabalhos futuros}
	\begin{itemize}
		\item Testar outros métodos de programação linear com Simplex, Dual Simplex;
		\item Realizar estudo na resolução das funções multiobjetivos deste trabalho empregando
		métodos de apoio a decisão multicritério;
		\item Avaliar o desempenho e viabilidade dos métodos de apoio a decisão, por meio de algorítimos.	
	\end{itemize}
\end{frame}
% ----------------- NOVO SLIDE --------------------------------
\section{Referências}

%\begin{frame}[allowframebreaks]{Referências}
%\bibliography{OUTROS_ARTIGOS,TCC_ALUNOS_DOUGLAS}
%\bibliography{Mendeley}

%\end{frame}

\begin{frame}[allowframebreaks]{Referências}
	%\framesubtitle{\TeX, \LaTeX, and Beamer}
	\begin{thebibliography}{9}
		\providecommand{\abntrefinfo}[3]{}
		\providecommand{\abntbstabout}[1]{}
		\abntbstabout{v-1.9.6 }
		
		\bibitem[ALBUQUERQUE, SANTOS e FILHO 2011]{Albuquerque2011}
		\abntrefinfo{ALBUQUERQUE, SANTOS e FILHO}{ALBUQUERQUE; SANTOS; FILHO}{2011}
		{ALBUQUERQUE, M.~C. et al. {Heuristica de Programa{\c{c}}{\~{a}}o
		  Matem{\'{a}}tica para o Problema de Fluxo Multiproduto Binario}. In:
		  \emph{Simp{\'{o}}sio Brasileiro de Pesquisa Operacional}. [S.l.: s.n.], 2011.
		  p. 2147--2156.}
		
		\bibitem[AMORIM 2006]{AMORIM2006}
		\abntrefinfo{AMORIM}{AMORIM}{2006}
		{AMORIM, E. d.~A.
		\emph{{Fluxo de pot{\^{e}}ncia {\'{o}}timo em sistemas multimercados
		  atrav{\'{e}}s de um algor{\'{i}}tmo evolutivo multiobjetivo}}.
		Tese (Doutorado) --- Universidade Estadual Paulista, 2006.}
		
		\bibitem[AZUMA 2011]{AZUMA2011}
		\abntrefinfo{AZUMA}{AZUMA}{2011}
		{AZUMA, R.~M. et~al.
		\emph{{Otimiza{\c{c}}{\~{a}}o multiobjetivo em problema de estoque e roteamento
		  gerenciados pelo fornecedor}}.
		1--113~p. Tese (Mestrado) --- Universidade Estadual de Campinas, 2011.}
		
		\bibitem[CORRAR, THE{\'{O}}PHILO e BERGMANN 2004]{CORRAR2004}
		\abntrefinfo{CORRAR, THE{\'{O}}PHILO e BERGMANN}{CORRAR; THE{\'{O}}PHILO;
		  BERGMANN}{2004}
		{CORRAR, L.~J. et al. \emph{{Pesquisa operacional para decis{\~{a}}o em
		  contabilidade e administra{\c{c}}{\~{a}}o: contabilometria}}. S{\~{a}}o
		  Paulo: Atlas, 2004.}
		
		\bibitem[CORREIA 2017]{CORREIA2017}
		\abntrefinfo{CORREIA}{CORREIA}{2017}
		{CORREIA, A. M.~P.
		Monografia, \emph{{An{\'{a}}lise de m{\'{e}}todos escalares aplicados a
		  problema multiobjetivos}}. [S.l.]: Universidade Estadual do Tocantins, 2017.
		  1--63~p.}
		
		\bibitem[COUTINHO 2008]{Coutinho2008}
		\abntrefinfo{COUTINHO}{COUTINHO}{2008}
		{COUTINHO, B.~C.
		\emph{{Proposta de Algoritmo H{\'{i}}brido para o Problema de Tarefas em
		  Ambientes Distribu{\'{i}}dos Homog{\^{e}}neos}}.
		Tese (Mestrado) --- Universidade Federal do Esp{\'{i}}rito Santo, 2008.}
		
		\bibitem[COUTINHO e {DE OLIVEIRA} 2009]{COUTINHO2009}
		\abntrefinfo{COUTINHO e {DE OLIVEIRA}}{COUTINHO; {DE OLIVEIRA}}{2009}
		{COUTINHO, B.~C.; {DE OLIVEIRA}, E.~S. {Fixa{\c{c}}{\~{a}}o de vari{\'{a}}veis
		  do modelo matem{\'{a}}tico aplicado na solu{\c{c}}{\~{a}}o do problema de
		  escalonamento de tarefas em ambientes distribu{\'{i}}dos homog{\^{e}}neos}.
		\emph{SBPO - Simp{\'{o}}sio Brasileiro de Pesquisa Operacional}, 2009.}
		
		\bibitem[ETCHEVERRY 2012]{ETCHEVERRY2012}
		\abntrefinfo{ETCHEVERRY}{ETCHEVERRY}{2012}
		{ETCHEVERRY, G.~V.
		\emph{{Programa{\c{c}}{\~{a}}o de tarefas em m{\'{a}}quinas paralelas
		  n{\~{a}}o-relacionadas com tempos de setup dependentes da sequ{\^{e}}ncia}}.
		1--54~p. Tese (Mestrado) --- Universidade Federal do Rio Grande do Sul, 2012.}
		
		\bibitem[HADDAD 2012]{HADDAD2012}
		\abntrefinfo{HADDAD}{HADDAD}{2012}
		{HADDAD, M.~N. \emph{{Algoritmos heur{\'{i}}sticos h{\'{i}}bridos para o
		  problema de sequenciamento em m{\'{a}}quinas paralelas n{\~{a}}o-relacionadas
		  com tempos de prepara{\c{c}}{\~{a}}o dependentes da sequ{\^{e}}ncia}}. 2012.}
		
		\bibitem[HILLIER e LIEBERMAN 2006]{HILLIER2006}
		\abntrefinfo{HILLIER e LIEBERMAN}{HILLIER; LIEBERMAN}{2006}
		{HILLIER, F.~S.; LIEBERMAN, G.~J. \emph{{Introdu{\c{c}}ao a Pesquisa
		  Operacional}}. 8. ed. S{\~{a}}o Paulo: McGraw-Hill Interamericana do Brasil
		  Ltda, 2006.
		ISBN 85-86804-68-1.}
		
		\bibitem[HILLIER e LIEBERMAN 2013]{HILLIER2013}
		\abntrefinfo{HILLIER e LIEBERMAN}{HILLIER; LIEBERMAN}{2013}
		{\underline{\ \ \ \ \ \ \ \ }. \emph{{Introdu{\c{c}}{\~{a}}o {\`{a}} pesquisa
		  operacional}}. [S.l.: s.n.], 2013.}
		
		\bibitem[KAWAMURA 2006]{KAWAMURA2006}
		\abntrefinfo{KAWAMURA}{KAWAMURA}{2006}
		{KAWAMURA, M.~S.
		\emph{{Aplica{\c{c}}{\~{a}}o do m{\'{e}}todo branch-and-bound na
		  programa{\c{c}}{\~{a}}o de tarefas em uma {\'{u}}nica m{\'{a}}quina com data
		  de entrega comum sob penalidades de adiantamento e atraso.}}
		Tese (Tese de Doutorado) --- Universidade de S{\~{a}}o Paulo, 2006.}
		
		\bibitem[MARTINS 2017]{MARTINS2017}
		\abntrefinfo{MARTINS}{MARTINS}{2017}
		{MARTINS, R.~C. \emph{{An{\'{a}}lise de algoritmos evolucion{\'{a}}rios na
		  resolu{\c{c}}{\~{a}}o de fun{\c{c}}{\~{o}}es de benchmark irrestritas.}}
		  [S.l.]: Universidade Estadual do Tocantins, 2017.}
		
		\bibitem[MINEIRO 2007]{MINEIRO2007}
		\abntrefinfo{MINEIRO}{MINEIRO}{2007}
		{MINEIRO, A.
		\emph{{Aplica{\c{c}}{\~{a}}o de programa{\c{c}}{\~{a}}o n{\~{a}}o-linear como
		  ferramenta de aux{\'{i}}lio {\`{a}} tomada de decis{\~{a}}o na gest{\~{a}}o
		  de um clube de investimento}}.
		91~p. Tese (Mestrado) --- UNIFEI, Itajub{\'{a}}, 2007.}
		
		\bibitem[M{\"{U}}LLER e DIAS 2002]{MULLER2002}
		\abntrefinfo{M{\"{U}}LLER e DIAS}{M{\"{U}}LLER; DIAS}{2002}
		{M{\"{U}}LLER, F.~M.; DIAS, O.~B. {Algoritmo para o problema de
		  seq{\"{u}}enciamento em m{\'{a}}quinas paralelas n{\~{a}}o-relacionadas}.
		\emph{Revista Produ{\c{c}}{\~{a}}o}, v.~12, n.~2, p. 6--17, 2002.
		ISSN 0103-6513.}
		
		\bibitem[NARI{\~{N}}O, MARTHA e {DE MENEZES} 2014]{NARINO2014}
		\abntrefinfo{NARI{\~{N}}O, MARTHA e {DE MENEZES}}{NARI{\~{N}}O; MARTHA; {DE
		  MENEZES}}{2014}
		{NARI{\~{N}}O, G. A.~R. et al.
		\emph{{Otimiza{\c{c}}{\~{a}}o de risers em caten{\'{a}}ria com amortecedores
		  hidrodin{\^{a}}micos}}.
		18~p. Tese (Mestrado) --- PUC-Rio, 2014.}
		
		\bibitem[PACHAMANOVA e FABOZZI 2010]{PACHAMANOVA2010}
		\abntrefinfo{PACHAMANOVA e FABOZZI}{PACHAMANOVA; FABOZZI}{2010}
		{PACHAMANOVA, D.~A.; FABOZZI, F.~J. \emph{{Simulation and Optimization in
		  Finance: Modeling with MATLAB,@ RISK, or VBA}}. [S.l.]: John Wiley {\&} Sons,
		  2010. 787~p.
		ISBN 9780470371893.}
		
		\bibitem[QUEIROZ 2011]{QUEIROZ2011}
		\abntrefinfo{QUEIROZ}{QUEIROZ}{2011}
		{QUEIROZ, M.~M. de.
		\emph{{M{\'{e}}todos heur{\'{i}}sticos aplicados ao problema de
		  programa{\c{c}}{\~{a}}o da frota de navios PLVs}}.
		Tese (Doutorado) --- Universidade de S{\~{a}}o Paulo, 2011.}
		
		\bibitem[REIS, DUQUE e VILLELA 2010]{REIS2010}
		\abntrefinfo{REIS, DUQUE e VILLELA}{REIS; DUQUE; VILLELA}{2010}
		{REIS, D.~C. et al. {Problema da Aloca{\c{c}}{\~{a}}o de Monitores de Qualidade
		  de Energia El{\'{e}}trica em Redes de Transmiss{\~{a}}o}.
		\emph{XVIII Congresso Brasileiro de Autom{\'{a}}tica}, p.~1--6, 2010.}
		
		\bibitem[ROCHA 2006]{ROCHA2006}
		\abntrefinfo{ROCHA}{ROCHA}{2006}
		{ROCHA, P.~L.
		\emph{{Um problema de sequenciamento em m{\'{a}}quinas paralelas
		  n{\~{a}}o-relacionadas com tempos de prepara{\c{c}}{\~{a}}o dependentes de
		  m{\'{a}}quina e da sequ{\^{e}}ncia:: modelos e algoritmos exato}}.
		1--70~p. Tese (Mestrado) --- Universidade Federal de Minas Gerais, 2006.}
		
		\bibitem[SECCHI 2015]{SECCHI2015}
		\abntrefinfo{SECCHI}{SECCHI}{2015}
		{SECCHI, A.~R. \emph{{COQ-897 - Otimiza{\c{c}}{\~{a}}o de Processos}}. Rio de
		  Janeiro: [s.n.], 2015.}
		
		\bibitem[SERAFINI, ANZANELLO e KAHMANN 2016]{SERAFINI2016}
		\abntrefinfo{SERAFINI, ANZANELLO e KAHMANN}{SERAFINI; ANZANELLO; KAHMANN}{2016}
		{SERAFINI, L. et al. {Heur{\'{i}}stica para minimiza{\c{c}}{\~{a}}o do atraso
		  total de tarefas baseada em curvas de aprendizado e aspectos
		  ergon{\^{o}}micos}.
		\emph{Revista Produ{\c{c}}{\~{a}}o Online}, Florian{\'{o}}polis, p. 550--574,
		  2016.}
		
		\bibitem[SILVA 2008]{SILVA2008}
		\abntrefinfo{SILVA}{SILVA}{2008}
		{SILVA, V.~V. et~al. {Uma heur{\'{i}}stica h{\'{i}}brida para o problema de
		  escalonamento de tarefas peri{\'{o}}dico em m{\'{a}}quinas paralelas}.
		\emph{XXVIII Encontro Nacional de Engenharia de Produ{\c{c}}{\~{a}}o}, p.
		  1--13, 2008.}
		
		\bibitem[XAVIER 2003]{XAVIER2003}
		\abntrefinfo{XAVIER}{XAVIER}{2003}
		{XAVIER, E.~C. et~al.
		\emph{{Algoritmos de aproxima{\c{c}}{\~{a}}o para problemas de escalonamento de
		  tarefas em maquinas}}.
		1--148~p. Tese (Mestrado) --- Unicamp, 2003.}
		
		\bibitem[YUNES 2000]{YUNES2000}
		\abntrefinfo{YUNES}{YUNES}{2000}
		{YUNES, T.~H. et~al.
		\emph{{Problemas de escalonamento no transporte coletivo:
		  programa{\c{c}}{\~{a}}o por restri{\c{c}}{\~{o}}es e outras tecnicas}}.
		Tese (Mestrado) --- Unicamp, 2000.}
		
	\end{thebibliography}
  \end{frame}
% ----------------- FIM DO DOCUMENTO -----------------------------------------
\end{document}
